\chapter{Cifras Clássicas}
\label{cha:cifras-classicas}

A internet é um meio de comunicação inerentemente promíscuo.
As partes que se comunicam pela rede não têm controle sobre os caminhos que suas mensagens percorrem.
Essa característica, entretanto, não é exclusiva da era digital.
No século XVIII, por exemplo, toda correspondência que passava pelo serviço de correios de Viena, na Áustria, era direcionada para um escritório chamado de ``black chamber''.
Lá, o selo da carta era derretido, o conteúdo copiado, o selo recolocado e a carta reenviada ao destinatário.
Todo esse processo durava cerca de três horas para evitar atrasos na entrega.
Tal prática não era isolada; todas as potências europeias da época operavam suas próprias ``black chambers''.

Com a invenção do telégrafo e do rádio, a espionagem se tornou ainda mais sofisticada.
O telégrafo permitiu a instalação de grampos nas linhas de comunicação, enquanto o rádio possibilitou a interceptação de transmissões sem fio \cite{Kahn96}.

Essa capacidade de monitoramento e interceptação persiste na era digital, onde as comunicações pela internet podem ser facilmente observadas por diversos atores, sejam governamentais ou privados.
Portanto, a necessidade de métodos robustos de criptografia nunca foi tão crítica.

Para estudar a segurança da comunicação vamos adotar o seguinte modelo.
Duas partes, o remetente e o destinatário, desejam se comunicar.
Tradicionalmente, chamamos o remetente de Alice e o destinatário de Bob.
O uso dos nomes Alice e Bob é comum em todos os livros de criptografia, provavelmente porque suas letras iniciais são as duas primeiras letras do alfabeto: A e B.

Nossa principal suposição é que o canal de comunicação entre Alice e Bob é inseguro.
Isso significa que assumimos que terceiros, a quem chamamos de Eva, podem observar as mensagens que trafegam pelo canal.
O nome Eva vem do inglês ``Eve'', derivado de ``eavesdrop'', que significa ``bisbilhotar''.

Essa suposição é frequentemente referida como a ``hipótese da comunicação hacker''.
Para os propósitos deste curso, sempre trabalharemos com essa hipótese.

A {\em criptografia} (do grego ``escrita secreta'') é a prática e o estudo de técnicas para garantir a comunicação segura na presença de terceiros, conhecidos como {\em adversários}.
O nosso primeiro desafio neste curso é apresentar sistemas de comunicação que garantam a {\em confidencialidade}.
Em outras palavras, qualquer mensagem enviada de Alice para Bob deve ser compreensível apenas para Alice e Bob e incompreensível para Eva, o adversário.

\begin{center}
  \begin{tikzpicture}[node distance=2cm,auto,>=latex]
    \node[draw=none] (alice) {\stickfigure};
    \node[above of=alice, node distance=1.2cm] {Alice};
    \node[draw=none] (bob) at (10,0) {\stickfigure};
    \node[above of=bob, node distance=1.2cm] {Bob};
\node (eva) at (5,2) {Eva};
\draw[->] (alice) -> node[above]{mensagem} (bob);
\path[->] (eva) edge (5,.5);
\end{tikzpicture}
\end{center}

A importância da comunicação confidencial entre civis tem se tornado cada vez mais urgente, mas no meio militar suas origens são ainda mais antigas.
Suetônio (69-141 d.C.), um historiador e biógrafo romano, descreveu, cerca de dois mil anos atrás, como o imperador Júlio César (100 a.C. - 44 a.C.) escrevia mensagens confidenciais:

\begin{quote}
  ``Se ele tinha qualquer coisa confidencial a dizer, ele escrevia cifrado, isto é, mudando a ordem das letras do alfabeto, para que nenhuma palavra pudesse ser compreendida.
  Se alguém deseja decifrar a mensagem e entender seu significado, deve substituir a quarta letra do alfabeto, a saber 'D', por 'A', e assim por diante com as outras.''
\end{quote}

O esquema que chamaremos de cifra de César é ilustrado pelo seguinte exemplo:

\begin{verbatim}
Mensagem: transparencia
Cifra:    XUDQVSDUHQFLD
\end{verbatim}

Como descrito por Suetônio, a regra para encriptar uma mensagem de Júlio César consistia em substituir cada letra da mensagem por aquela que está três posições à frente na ordem alfabética.
Para descriptografar a cifra, substitui-se cada letra por aquela que está três posições atrás.

Aqueles que, como César, conheciam ou descobriam essa regra eram capazes de entender as mensagens.
Para todos os outros, a mensagem parecia apenas uma sequência sem sentido de letras.
Esse método simples, embora engenhoso para a época, tinha uma grande vulnerabilidade:
uma vez que a regra de criptografia era conhecida, qualquer um podia decifrar a mensagem facilmente.
Em outras palavras, o segredo é sua própria regra.

Embora técnicas de criptografia e criptoanálise existam desde o Império Romano, foi com o advento do telégrafo e sua capacidade de comunicação eficiente que o campo se estruturou.
No fim do século XIX, um militar holandês chamado Auguste Kerckhoffs estabeleceu um princípio fundamental para as cifras, que ficou conhecido como o {\em princípio de Kerckhoffs}.

Esse princípio estabelece que a regra usada para criptografar uma mensagem, mesmo que codificada em um mecanismo, não deve ser um segredo.
Ou seja, não deve ser um problema se esse mecanismo cair nas mãos do adversário ou que a regra seja descoberta.

Outros criptógrafos importantes reafirmaram esse princípio de maneiras diferentes.
Claude Shannon, considerado o fundador da teoria da informação, dizia que devemos sempre assumir que ``o inimigo conhece o sistema''.
Whitfield Diffie, um dos pioneiros da criptografia moderna, afirmou que ``um segredo que não pode ser rapidamente modificado deve ser interpretado como uma vulnerabilidade''.

Ou seja, em uma comunicação confidencial, as partes devem compartilhar algo que possa ser facilmente modificado caso necessário.
Esse segredo compartilhado é chamado de {\em chave} da comunicação e assumimos que ela é a única parte sigilosa do sistema.

Trazendo o debate para uma discussão mais moderna, o sigilo do código-fonte de um sistema não deve, em hipótese alguma, ser o que garante sua segurança.
A segurança deve depender apenas da chave secreta, não do algoritmo ou do código-fonte.

O modelo de {\em criptografia simétrica} pode ser descrito da seguinte maneira:
Alice (a remetente) usa um algoritmo público ($E$) que, dada uma chave ($k$), transforma uma mensagem ($m$) em um texto incompreensível chamado de cifra ($c$).
A cifra é então enviada a Bob (o destinatário) através de um meio assumidamente inseguro (hipótese da comunicação hacker).
Bob, por sua vez, usa a mesma chave em um outro algoritmo (D) para recuperar a mensagem original a partir da cifra.

\begin{center}
\begin{tikzpicture}[node distance=2cm,auto,>=latex]
\node (alice) at (0, 2){Alice};
\node (bob) at (10, 2) {Bob};
\node (eva) at (5, 2) {Eva};

\node (m1) at (0,1) {$m$};
\node (k1) at (0,-1) {$k$};
\node (E)  at (2,0) {$E(k,m) = c$};
\node (D)  at (8,0) {$D(k,c) = m$};
\node (k2) at (10,-1) {$k$};
\node (m2) at (10,1) {$m$};

\path[->] (eva) edge (5,1);
\draw[->] (m1) -> (E);
\draw[->] (k1) -> (E);
\draw[->] (D) -> (m2);
\draw[->] (k2) -> (D);
\draw[->] (E) -> node[above]{$c$} (D);
\end{tikzpicture}
\end{center}

Este modelo é chamado de simétrico porque assume que ambas as partes, Alice e Bob, compartilham a mesma chave.
Mais adiante, veremos um outro modelo chamado assimétrico.

Um sistema de criptografia simétrico é formado por três algoritmos:
\begin{itemize}
\item $Gen$: o algoritmo que gera uma chave secreta.
\item $E$: o algoritmo que criptografa, ou seja, transforma uma mensagem em uma cifra.
\item $D$: o algoritmo que descriptografa, ou seja, recupera a mensagem a partir de uma cifra.
\end{itemize}

Considere que temos um sistema $\Pi$ formado por três algoritmos como esses.
Digamos que uma mensagem $m$ foi criptografada usando o algoritmo $E$ e a chave $k$, produzindo a cifra $c$ (escrevemos $E(k,m) = c$).
O sistema é considerado {\em correto} se, ao descriptografarmos $c$ com o algoritmo $D$ usando a mesma chave $k$, recuperamos a mensagem original $m$ (ou seja, $D(k,c) = m$).

\section{Cifra de Deslocamento}
\label{sec:cifra-deslocamento}

O que chamamos na seção anterior de ``cifra de César'' não deve ser propriamente considerado um sistema de criptografia, pois não possui uma chave.
No entanto, é possível e simples adaptar esse esquema para incorporar uma chave.

Para fazer isso, em vez de deslocar as letras sempre três casas para frente, vamos assumir que foi sorteado previamente um número $k$ entre $0$ e $25$.
Ou seja, nosso algoritmo $Gen$ sorteia um número de $0$ a $25$
Esse número será a chave da comunicação, e assumiremos que ambas as partes a compartilham.

O mecanismo para criptografar uma mensagem ($E$) será deslocar cada letra $k$ posições para a direita.
Para descriptografar ($D$), basta deslocar cada letra as mesmas $k$ posições para a esquerda.

\begin{example}
  Digamos que o algoritmo $Gen$ sorteie a chave $k = 3$:

  A Tabela \ref{table:cesar_shift} mostra a posição original de cada letra e a nova posição após o deslocamento de 3 posições.
  Cada linha da tabela indica a letra original, sua posição no alfabeto, a nova posição após o deslocamento e a letra resultante após o deslocamento.

  
  \begin{table}[h!]
    \centering
    \begin{tabular}{|c|c|c|c|}
      \hline
      \multicolumn{2}{|c|}{\textbf{Original}} & \multicolumn{2}{c|}{\textbf{Deslocada}} \\
      \hline
      \textbf{Letra} & \textbf{Posição} & \textbf{Posição} & \textbf{Letra} \\
      \hline
      \texttt{a} & 0 & 3 & \texttt{D} \\
      \hline
      \texttt{b} & 1 & 4 & \texttt{E} \\
      \hline
      \texttt{c} & 2 & 5 & \texttt{F} \\
      \hline
      \texttt{d} & 3 & 6 & \texttt{G} \\
      \hline
      \texttt{e} & 4 & 7 & \texttt{H} \\
      \hline
      \texttt{f} & 5 & 8 & \texttt{I} \\
      \hline
      \texttt{g} & 6 & 9 & \texttt{J} \\
      \hline
      \texttt{h} & 7 & 10 & \texttt{L} \\
      \hline
      \texttt{i} & 8 & 11 & \texttt{M} \\
      \hline
      \texttt{j} & 9 & 12 & \texttt{N} \\
      \hline
      \texttt{l} & 10 & 13 & \texttt{O} \\
      \hline
      \texttt{m} & 11 & 14 & \texttt{P} \\
      \hline
      \texttt{n} & 12 & 15 & \texttt{Q} \\
      \hline
      \texttt{o} & 13 & 16 & \texttt{R} \\
      \hline
      \texttt{p} & 14 & 17 & \texttt{S} \\
      \hline
      \texttt{q} & 15 & 18 & \texttt{T} \\
      \hline
      \texttt{r} & 16 & 19 & \texttt{U} \\
      \hline
      \texttt{s} & 17 & 20 & \texttt{V} \\
      \hline
      \texttt{t} & 18 & 21 & \texttt{X} \\
      \hline
      \texttt{u} & 19 & 22 & \texttt{Z} \\
      \hline
      \texttt{v} & 20 & 23 & \texttt{A} \\
      \hline
      \texttt{x} & 21 & 0 & \texttt{B} \\
      \hline
      \texttt{z} & 22 & 1 & \texttt{C} \\
      \hline
    \end{tabular}
    \caption{Tabela de deslocamento para $k=3$}
    \label{table:cesar_shift}
  \end{table}

Para calcular a nova posição de cada letra após o deslocamento, usamos a {\em aritmética modular}.
A aritmética modular é uma maneira de lidar com valores que ``reiniciam'' após atingirem um certo valor, similar ao funcionamento de um relógio.
No contexto da cifra de deslocamento, utilizamos a aritmética modular para assegurar que, ao deslocar uma letra, continuamos dentro dos limites do alfabeto.

Como estamos usando o alfabeto latino básico, que possui 23 letras, para obter a posição deslocada somamos $k$ à posição original módulo 23.
Ou seja, pegamos o resto da soma quando dividido por 23.  

Se a mensagem for {\tt transparencia} a cifra será {\tt XUDQVSDUHQFLD}.

  \begin{itemize}
  \item[] E($3$,{\tt transparencia}) =  {\tt XUDQVSDUHQFLD} 
  \item[] D($3$,{\tt XUDQVSDUHQFLD}) =  {\tt transparencia} 
  \end{itemize}
\end{example}

A criptoanálise da cifra de deslocamento revela que ela não é segura, principalmente porque a quantidade de chaves existentes é pequena.
{\em Criptoanálise} é a prática de analisar e decifrar sistemas criptográficos com o objetivo de encontrar vulnerabilidades ou descobrir o conteúdo de mensagens cifradas sem conhecer a chave secreta.
Chamamos de {\em universo das chaves} o conjunto que abrange todas as chaves possíveis em um sistema de criptografia.
No caso da cifra de deslocamento, a chave é um número entre 0 e 22, resultando em apenas 23 chaves possíveis.
Este pequeno universo de chaves torna o sistema vulnerável a ataques de força bruta.

Um {\em ataque de força bruta} envolve testar todas as chaves possíveis até encontrar a correta.
Dado que existem apenas 23 chaves possíveis na cifra de deslocamento, um atacante pode simplesmente tentar cada uma delas.
Em média, precisaríamos testar $\frac{23}{2} =11,5$ chaves até encontrar a correta.
O valor esperado da quantidade de tentativas necessárias para encontrar a chave correta é sempre metade do tamanho do universo das chaves.

Embora todos os sistemas de criptografia estejam teoricamente sujeitos a ataques de força bruta, a viabilidade prática do ataque depende do tamanho do universo de chaves.
Em sistemas com um grande universo de chaves, um ataque de força bruta se torna impraticável devido ao tempo e aos recursos computacionais necessários.

\begin{example}
  Muitos dos roteadores modernos possuem um mecanismo chamado WPS (Wi-Fi Protected Setup), que supostamente simplifica o processo de conexão, especialmente na configuração do hardware.
  O WPS permite que um usuário se conecte remotamente e sem fio ao roteador, desde que possua um PIN (Personal Identification Number).
  Esse PIN é uma sequência de oito dígitos de 0 a 9, o que significa que o universo das chaves é $10^8 \approx 2^{27}$.

  Para entender melhor, o universo das chaves, $10^8$, significa que existem 100 milhões de combinações possíveis.
  Em termos binários, isso é aproximadamente $2^{27}$, que é um número relativamente pequeno para os padrões de segurança modernos.

  Devido ao tamanho limitado do universo das chaves, um ataque de força bruta é viável.
  Um ataque de força bruta envolve tentar todas as combinações possíveis de PIN até encontrar a correta.
  Neste contexto, um atacante pode testar sistematicamente cada um dos 100 milhões de PINs possíveis.
  Dependendo da velocidade da conexão e da capacidade de processamento do atacante, este processo pode levar entre 4 e 8 horas para ser concluído.

  Esse tempo relativamente curto para realizar um ataque de força bruta expõe uma significativa vulnerabilidade no mecanismo WPS.
  Uma vez que um atacante consegue encontrar o PIN correto, ele pode acessar a rede Wi-Fi, comprometendo a segurança e a privacidade dos dados transmitidos.

  Portanto, enquanto o WPS pode facilitar a configuração inicial do roteador, ele também introduz um risco considerável de segurança.
  Para mitigar esse risco, é recomendado desativar o WPS e utilizar métodos de configuração de segurança mais robustos, como a configuração manual de uma senha Wi-Fi forte e a utilização de protocolos de segurança avançados como WPA3.
\end{example}

Como em geral uma chave é uma sequência aleatória de bits, uma chave de $n$ bits corresponde a um universo de $2^n$ chaves.
O tamanho da chave seguro é algo que depende do avanço tecnológico.
Nos anos 70, o padrão para criptografia usava chaves com 56 bits, que eram consideradas seguras na época.
No entanto, em 1998, uma chave deste tamanho foi quebrada em um ataque de força bruta, demonstrando que a evolução tecnológica havia tornado esse tamanho de chave vulnerável.

Apesar disso, um universo de chaves de 80 bits, é tão grande que mesmo os avanços tecnológicos não devem comprometer sua segurança.
Como veremos, hoje em dia, os sistemas costumam ter chaves com pelo menos 128 bits, o que torna um ataque força bruta contra esses sistemas inviável. 

\section{Cifra de Substituição}
\label{sec:cifra-monoalfabetica}

Em 1567, a residência da rainha da Escócia, foi destruída por uma explosão, resultando na morte do então rei, que era seu primo.
O principal suspeito do assassinato foi dispensado da pena e, surpreendentemente, casou-se com Mary no mês seguinte.
Este evento tumultuoso levou à prisão de Mary na Inglaterra.

Mary subiu ao trono da Escócia em 1542.
Muitos católicos consideravam Mary a legítima herdeira do trono inglês, que estava ocupado pela protestante Elizabeth I, que reinava na Inglaterra desde 1558.
Durante seu tempo na prisão, Mary conspirou com seus aliados para assassinar Elizabeth e reivindicar o trono.


Em 1587, Mary foi executada pelo que ficou conhecido como a Conspiração de Babington.
A principal prova usada para sua condenação foi uma troca de cartas cifradas que foram interceptadas e decifradas pelas autoridades inglesas \cite{Singh04}.

A cifra usada pelos conspiradores é conhecida hoje como {\em cifra de substituição} ou {\em cifra monoalfabética}.
Neste tipo de criptografia, cada letra ou par de letras é substituída por um símbolo, que pode ser inclusive outra letra.

A chave de uma cifra de substituição é uma tabela que associa cada letra a um símbolo.
Para que a cifra funcione corretamente, a quantidade de símbolos deve coincidir com a quantidade de letras no alfabeto.
Isso significa que podemos substituir cada letra por outra letra sem perder nenhuma informação.

Em outras palavras, a chave de uma cifra de substituição é uma permutação do alfabeto.
Cada letra do alfabeto é substituída por outra letra de acordo com esta tabela de permutação.

\begin{example}

Vamos considerar que temos a seguinte Tabela de substituição \ref{table:substitution_cipher}.
  
\begin{table}[h!]
\centering
\begin{tabular}{|c|c|}
\hline
\textbf{Letra Original} & \textbf{Letra Substituída} \\
\hline
\texttt{a} & \texttt{Z} \\
\hline
\texttt{b} & \texttt{E} \\
\hline
\texttt{c} & \texttt{B} \\
\hline
\texttt{d} & \texttt{R} \\
\hline
\texttt{e} & \texttt{A} \\
\hline
\texttt{f} & \texttt{S} \\
\hline
\texttt{g} & \texttt{C} \\
\hline
\texttt{h} & \texttt{D} \\
\hline
\texttt{i} & \texttt{F} \\
\hline
\texttt{j} & \texttt{G} \\
\hline
\texttt{k} & \texttt{H} \\
\hline
\texttt{l} & \texttt{I} \\
\hline
\texttt{m} & \texttt{J} \\
\hline
\texttt{n} & \texttt{K} \\
\hline
\texttt{o} & \texttt{L} \\
\hline
\texttt{p} & \texttt{M} \\
\hline
\texttt{q} & \texttt{N} \\
\hline
\texttt{r} & \texttt{O} \\
\hline
\texttt{s} & \texttt{P} \\
\hline
\texttt{t} & \texttt{Q} \\
\hline
\texttt{u} & \texttt{T} \\
\hline
\texttt{v} & \texttt{U} \\
\hline
\texttt{w} & \texttt{V} \\
\hline
\texttt{x} & \texttt{W} \\
\hline
\texttt{y} & \texttt{X} \\
\hline
\texttt{z} & \texttt{Y} \\
\hline
\end{tabular}
\caption{Tabela de permutação da cifra de substituição}
\label{table:substitution_cipher}
\end{table}

A partir desta chave, podemos criptografar textos substituindo cada letra pela letra correspondente na chave.
Por exemplo, usando a tabela de permutação fornecida, a letra {\tt a} seria substituída por {\tt Z}, a letra {\tt b} por {\tt E}, e assim por diante.
Este processo transforma o texto original em um texto cifrado, onde cada letra do alfabeto original é substituída por sua contraparte definida na chave de permutação.

\begin{verbatim}
  Mensagem: transparencia
  Cifra:    QOZKPMZOAKBFZ
\end{verbatim}

Para descriptografar o texto cifrado, basta reverter o processo.
Isso significa substituir cada letra do texto cifrado pela letra original do alfabeto.
Usando novamente a tabela de permutação, se encontrarmos a letra {\tt Z} no texto cifrado, substituímos de volta por {\tt a}, se encontrarmos {\tt E}, substituímos por {\tt b}, e assim por diante.
Este processo inverso reconstrói o texto original a partir do texto cifrado.
\end{example}

O desfecho da história da Conspiração de Babington sugere que a cifra monoalfabética não é muito segura.
No entanto, até o desenvolvimento das primeiras máquinas de criptografar, versões das cifras monoalfabéticas eram as mais populares no mundo todo.

Um exemplo interessante é que, nos anos 70, a Editora Abril publicou no Brasil o famoso ``Manual do Escoteiro Mirim'' da Disney, que apresentava uma cifra monoalfabética.
Este manual era um sucesso entre as crianças e ensinava técnicas básicas de criptografia usando cifras de substituição simples.

Em 2017 o curioso caso do desaparecimento de um rapaz no Acre viralizou quando seus familiares revelaram que, no seu quarto, havia uma coleção de livros que ele havia escrito de maneira criptografada.
Após investigações, foi descoberto que o rapaz usara uma cifra de substituição.

A cifra de substituição é interessante porque seu universo de chaves é extremamente vasto.
O universo das chaves da cifra de substituição é o conjunto de todas as permutações possíveis do alfabeto.
Se estamos usando o alfabeto latino básico, com 23 letras, a análise combinatória nos indica que esse universo possui $23! \approx 4.10^{26} \approx 2^{88}$ combinações possíveis.
Este número é dezenas de milhares de vezes maior do que o número de grãos de areia em todas as praias da Terra, estimado em $10^{19}$ e dez vezes maior do que o número de estrelas no universo observável, que é estimado em cerca de $10^{22}$.

Este número colossal torna um ataque de força bruta impraticável, pois seria necessário testar um número inimaginável de chaves para encontrar a correta.

Apesar disso, como sabe qualquer pessoa que tenha jogado jogos como criptogramas, que são vendidos pelas famosas revistas Coquetel, é relativamente fácil quebrar uma cifra de substituição.
Isso porque esse sistema é vulnerável a um outro tipo de ataque, o ataque de frequência.

Um {\em ataque de frequência} explora as características estatísticas do texto original.
Em qualquer idioma, certas letras aparecem com mais frequência do que outras.
Por exemplo, em textos em português, as letras {\tt a}, {\tt e} e {\tt o} são muito comuns, enquanto letras como {\tt x}, {\tt z} e {\tt q} são usadas com menos frequência.
Um criptoanalista pode analisar a frequência das letras no texto cifrado e compará-las com as frequências conhecidas do idioma para fazer suposições educadas sobre quais letras correspondem a quais.

Para piorar -- ou melhorar dependendo da perspectiva -- na maioria das línguas há dígrafos particulares.
Por exemplo, no português, dois símbolos repetidos provavelmente representam o {\tt r} ou o {\tt s}, e o {\tt h} quase sempre vem depois do {\tt l}, do {\tt c} ou do {\tt n}.
Se o texto a ser decifrado for suficientemente longo, essas pistas podem ser suficientes para quebrar a cifra.

No seguinte trecho de ``O escaravelho de ouro'' de Edgar Allan Poe a personagem descreve essa técnica que ela utilizou para decifrar um texto em inglês:

\begin{quote}
``Ora, no inglês, a letra que ocorre com mais frequência é a letra {\tt e}.
Depois dela, a sucessão é: {\tt a o i d h n r s t u y c f g l m w b k p q x z}.
O e prevalece de tal maneira que quase nunca se vê uma frase isolada em que ele não seja predominante.
Aqui nós temos, portanto, bem no início, uma base que permite mais do que um mero palpite.
O uso que se pode fazer da tabela é óbvio, mas, neste criptograma em particular, não precisamos nos valer dela por inteiro.
Como nosso caractere dominante é o {\tt 8}, começaremos assumindo que este é o {\tt e} do alfabeto normal. (...)''
\end{quote}

Em português, faz sentido separar as letras em cinco blocos, com frequência de ocorrência decrescente:
\begin{enumerate}
\item {\tt a}, {\tt e} e {\tt o}
\item {\tt s}, {\tt r} e {\tt i}
\item {\tt n}, {\tt d}, {\tt m}, {\tt u}, {\tt t} e {\tt c}
\item {\tt l}, {\tt p}, {\tt v}, {\tt g}, {\tt h}, {\tt q}, {\tt b} e {\tt f}
\item {\tt z}, {\tt j}, {\tt x}, {\tt k}, {\tt w} e {\tt y}
\end{enumerate}

Esses blocos refletem a probabilidade de ocorrência das letras em textos em português, com as letras dos primeiros blocos aparecendo com mais frequência do que as dos últimos blocos.
Essa distribuição permite que um criptoanalista faça suposições informadas sobre quais letras no texto cifrado correspondem a quais letras no texto original, facilitando a quebra da cifra.
Foi uma análise assim que permitiu que o governo inglês decifrasse os textos da rainha Mary da Escócia, o que acabou lhe custando a vida.

Esse tipo de análise pode ser feito manualmente ou, hoje em dia, com a ajuda de programas de computador, e é extremamente eficaz contra cifras de substituição simples.
A análise de frequência foi uma das primeiras técnicas de criptoanálise desenvolvidas e continua a ser uma ferramenta poderosa para quebrar cifras que não introduzem complexidade suficiente para mascarar as características estatísticas do texto original.

Portanto, embora o universo de chaves de uma cifra de substituição seja vasto, a simplicidade do algoritmo o torna vulnerável a técnicas de criptoanálise como a análise de frequência.

\section{Cifra de Vigenère}
\label{sec:cifra-de-vigenere}

A cifra de Vigenère foi criada no século XV e, ainda no começo do século XX, era considerada inquebrável.
Em 1868, o matemático e autor de ``Alice no País das Maravilhas'', Charles Lutwidge Dodgson (mais conhecido pelo pseudônimo Lewis Carroll), descreveu a cifra como ``inquebrável''.
Ainda em 1917, a cifra continuava a ser amplamente reconhecida, e um artigo da revista Scientific American a descreveu como ``impossível de decifrar''.

Essa percepção de invulnerabilidade decorre da sofisticação da cifra de Vigenère, que pertence a uma classe de cifras chamadas de {\em polialfabéticas}.
Diferente das cifras monoalfabéticas, que usam uma única substituição para todo o texto, as cifras polialfabéticas utilizam múltiplos alfabetos de substituição.
Isso significa que a mesma letra no texto original pode ser cifrada de diferentes maneiras, dependendo de sua posição e da chave utilizada.

Em poucas palavras, a cifra de Vigenère consiste em deslocar as letras do texto original em distâncias diferentes, determinadas por uma chave.
Em sua versão mais simples, a chave é uma palavra cuja primeira letra indica quantas casas devemos deslocar a primeira letra da mensagem, a segunda letra da chave indica quantas casas devemos deslocar a segunda letra, e assim por diante.
Quando a mensagem ultrapassa o tamanho da chave, a chave é repetida e o processo continua.

Para facilitar a conta na hora de criptografar e descriptografar, podemos usar uma tabela que indica para cada letra da mensagem e cada letra da chave qual é a letra correspondente na cifra.
Essa tabela é chamada de {\em tabula recta} e está representada na Figura \ref{tab:tabula-recta}.
A tabula recta é uma matriz que alinha os alfabetos de forma que cada linha representa um alfabeto deslocado.
Para cifrar, encontra-se a letra da mensagem na coluna da letra da chave correspondente, e a interseção dá a letra cifrada.

\begin{table}[htbp]
  \centering
  \begin{small}
  \setlength{\tabcolsep}{11pt} 
\begin{tabular}{|@{}c@{}|*{23}{@{}c@{}|}}
\hline
\textbf{} & \textbf{A} & \textbf{B} & \textbf{C} & \textbf{D} & \textbf{E} & \textbf{F} & \textbf{G} & \textbf{H} & \textbf{I} & \textbf{J} & \textbf{L} & \textbf{M} & \textbf{N} & \textbf{O} & \textbf{P} & \textbf{Q} & \textbf{R} & \textbf{S} & \textbf{T} & \textbf{U} & \textbf{V} & \textbf{X} & \textbf{Z} \\
\hline
\textbf{A} & A & B & C & D & E & F & G & H & I & J & L & M & N & O & P & Q & R & S & T & U & V & X & Z \\
\hline
\textbf{B} & B & C & D & E & F & G & H & I & J & L & M & N & O & P & Q & R & S & T & U & V & X & Z & A \\
\hline
\textbf{C} & C & D & E & F & G & H & I & J & L & M & N & O & P & Q & R & S & T & U & V & X & Z & A & B \\
\hline
\textbf{D} & D & E & F & G & H & I & J & L & M & N & O & P & Q & R & S & T & U & V & X & Z & A & B & C \\
\hline
\textbf{E} & E & F & G & H & I & J & L & M & N & O & P & Q & R & S & T & U & V & X & Z & A & B & C & D \\
\hline
\textbf{F} & F & G & H & I & J & L & M & N & O & P & Q & R & S & T & U & V & X & Z & A & B & C & D & E \\
\hline
\textbf{G} & G & H & I & J & L & M & N & O & P & Q & R & S & T & U & V & X & Z & A & B & C & D & E & F \\
\hline
\textbf{H} & H & I & J & L & M & N & O & P & Q & R & S & T & U & V & X & Z & A & B & C & D & E & F & G \\
\hline
\textbf{I} & I & J & L & M & N & O & P & Q & R & S & T & U & V & X & Z & A & B & C & D & E & F & G & H \\
\hline
\textbf{J} & J & L & M & N & O & P & Q & R & S & T & U & V & X & Z & A & B & C & D & E & F & G & H & I \\
\hline
\textbf{L} & L & M & N & O & P & Q & R & S & T & U & V & X & Z & A & B & C & D & E & F & G & H & I & J \\
\hline
\textbf{M} & M & N & O & P & Q & R & S & T & U & V & X & Z & A & B & C & D & E & F & G & H & I & J & L \\
\hline
\textbf{N} & N & O & P & Q & R & S & T & U & V & X & Z & A & B & C & D & E & F & G & H & I & J & L & M \\
\hline
\textbf{O} & O & P & Q & R & S & T & U & V & X & Z & A & B & C & D & E & F & G & H & I & J & L & M & N \\
\hline
\textbf{P} & P & Q & R & S & T & U & V & X & Z & A & B & C & D & E & F & G & H & I & J & L & M & N & O \\
\hline
\textbf{Q} & Q & R & S & T & U & V & X & Z & A & B & C & D & E & F & G & H & I & J & L & M & N & O & P \\
\hline
\textbf{R} & R & S & T & U & V & X & Z & A & B & C & D & E & F & G & H & I & J & L & M & N & O & P & Q \\
\hline
\textbf{S} & S & T & U & V & X & Z & A & B & C & D & E & F & G & H & I & J & L & M & N & O & P & Q & R \\
\hline
\textbf{T} & T & U & V & X & Z & A & B & C & D & E & F & G & H & I & J & L & M & N & O & P & Q & R & S \\
\hline
\textbf{U} & U & V & X & Z & A & B & C & D & E & F & G & H & I & J & L & M & N & O & P & Q & R & S & T \\
\hline
\textbf{V} & V & X & Z & A & B & C & D & E & F & G & H & I & J & L & M & N & O & P & Q & R & S & T & U \\
\hline
\textbf{X} & X & Z & A & B & C & D & E & F & G & H & I & J & L & M & N & O & P & Q & R & S & T & U & V \\
\hline
\textbf{Z} & Z & A & B & C & D & E & F & G & H & I & J & L & M & N & O & P & Q & R & S & T & U & V & X \\
\hline
\end{tabular}
\end{small}
\caption{Tabula Recta}
\label{tab:tabula-recta}
\end{table}


\begin{example}
  Considere a seguinte mensagem criptografada com a chave {\tt senha} usando a cifra de Vigenère:

\begin{verbatim}
Mensagem: transparencia
Chave:    senhasenhasen
Cifra:    LVNUSHEELNUMN
\end{verbatim}
\end{example}

Apesar da fama de ``inquebrável'' que a cifra de Vigenère ostentou até o começo do século XX, desde a metade do século anterior já eram conhecidos métodos de criptoanálise capazes de derrotar esse tipo de cifra.

Em 1854, John Hall Brock Thwaites submeteu um texto cifrado utilizando uma cifra supostamente por ele inventada.
Charles Babbage, o inventor das máquinas que precederam o computador moderno, mostrou que no fundo a cifra de Thwaites era equivalente à cifra de Vigenère.
Após ser desafiado, Babbage decifrou uma mensagem criptografada por Thwaites duas vezes com chaves diferentes.

Em 1863, Friedrich Kasiski formalizou um ataque contra a cifra de Vigenère que ficou conhecido como {\em teste de Kasiski}.
O ataque considera o fato de que a chave se repete com uma frequência fixa e, portanto, há uma probabilidade de produzir padrões reconhecíveis.

\begin{example}
  Considere o exemplo extraído da Wikipédia:
  
\begin{verbatim}
Mensagem: cryptoisshortforcryptography
Chave:    abcdabcdabcdabcdabcdabcdabcd
Cifra:    CSASTPKVSIQUTGQUCSASTPIUAQJB
\end{verbatim}
\end{example}

Note que o padrão {\tt CSASTP} se repete na cifra.
Isso ocorre porque o prefixo ``crypto'' foi criptografado com a mesma chave.
Em um texto suficientemente longo, um padrão como este fatalmente ocorrerá.
Uma vez encontrado, é calculada a distância entre as repetições.
Neste caso, a distância é 16, o que significa que o tamanho da chave deve ser um divisor de 16 (2, 4, 8 ou 16).
Com esta informação, podemos aplicar um ataque de frequência nos caracteres de 2 a 2, de 4 a 4, de 8 a 8 e de 16 a 16.

Embora a cifra de Vigenère seja significativamente mais complexa do que as cifras monoalfabéticas, ela não é invulnerável.
Técnicas avançadas de criptoanálise, desenvolvidas ao longo do tempo, eventualmente permitiram a quebra dessa cifra.
Por exemplo, o método de Kasiski e a análise de frequência foram ferramentas essenciais na criptoanálise da cifra de Vigenère.

\section{Máquinas de Criptografar}
\label{sec:maquinas}

No final da primeira década do século XX, foram inventadas as primeiras máquinas de criptografar.
A componente principal dessas {\em máquinas eletromecânicas} é um conjunto de {\em rotores}.
A configuração inicial dos rotores contém a chave da criptografia.
Cada vez que o operador pressiona uma tecla, os rotores embaralham as letras.
Dessa forma, essas máquinas rotoras se comportam como uma sofisticada cifra polialfabética.

Para descriptografar a mensagem, o operador precisa ajustar a máquina em modo de descriptografia, configurar a máquina com a chave secreta inicial e digitar o texto cifrado.
A máquina então se rearranja para produzir o texto original quando digitado.

Essas máquinas foram um grande avanço na criptografia, proporcionando um nível de segurança muito maior do que as cifras anteriores.
O uso de rotores e a capacidade de alterar a configuração inicial significavam que cada letra do texto original poderia ser cifrada de várias maneiras diferentes, dependendo da posição dos rotores, tornando a criptoanálise extremamente difícil.

As máquinas rotoras mais conhecidas são da série {\em Enigma}.
Elas foram criadas por um inventor alemão no final da Primeira Guerra Mundial e versões mais modernas foram extensamente usadas durante a Segunda Guerra pelo exército nazista.
As versões mais simples da máquina possuíam três rotores capazes de gerar $26^3 \approx 17.500$ possíveis configurações iniciais.
Além disso, era possível trocar a ordem dos rotores, multiplicando por 6 o número de combinações possíveis, chegando a um total de cerca de 105 mil possibilidades.
A versão utilizada pelo exército nazista, porém, permitia cerca de 150 trilhões de possibilidades.

Em 1939, Alan Turing, matemático britânico e pioneiro da computação, desenvolveu uma máquina eletromecânica chamada {\em Bombe}, capaz de decifrar algumas cifras de máquinas Enigma com 3 rotores.
Posteriormente, a Bombe foi melhorada para decifrar mensagens de máquinas Enigma mais sofisticadas.
Este trabalho de decifração foi crucial para os esforços dos Aliados durante a Segunda Guerra Mundial e marcou um avanço significativo na criptoanálise.

A história da computação esbarra na história da criptografia neste ponto.
Poucos anos antes da guerra, Alan Turing demonstrara que a satisfatibilidade da lógica de primeira ordem é um problema indecidível.
Para tanto, ele propôs um modelo computacional que hoje chamamos de Máquinas de Turing.
Diferente dos modelos computacionais anteriores, como o cálculo lambda de Church ou as funções recursivas de Gödel, o modelo de Turing era intuitivo.
Além disso, Turing mostrou que era possível construir com seu modelo uma Máquina Universal capaz de simular qualquer outra Máquina de Turing.
Esse resultado magnífico é o que dá origem à computação moderna.

O primeiro modelo de computador desenvolvido por Turing e sua equipe em Bletchley Park foi batizado de {\em Colossus} e tinha como principal propósito quebrar outra cifra usada pelos nazistas durante a guerra, a {\em cifra de Lorenz}.
A cifra de Lorenz é uma versão do que estudaremos com o nome de {\em cifra de fluxo}.

Para decifrar os códigos das máquinas Enigma e da cifra de Lorenz, os ingleses tiveram que contar não apenas com o texto criptografado que interceptavam sem grandes dificuldades, mas também com uma série de cifras cujas mensagens eles conheciam previamente.
Veremos mais para frente a importância desta informação.

A capacidade dos aliados de decifrar as mensagens de seus adversários foi central para sua vitória na guerra.
O trabalho de Turing e de sua equipe em Bletchley Park, utilizando máquinas como a Bombe e o Colossus, foi fundamental para quebrar as cifras alemãs e garantiu uma vantagem estratégica crucial aos Aliados.

O começo do século XX marcou o surgimento das primeiras máquinas de criptografar e das primeiras máquinas de criptoanálise.
Na metade do século, começaram a surgir os primeiros computadores.
Nos anos 70, a comunicação seria revolucionada pelo advento da internet, mas antes disso já ficara claro que era necessário compreender melhor o que faz uma cifra ser segura.

\section{Exercício}
\label{sec:exercicio}

\begin{exercicio}
  Considere a seguinte mensagem:
  \begin{verbatim}
    privacidadepublicatranparenciaprivada
  \end{verbatim}
  \begin{itemize}
  \item Criptografe essa mensagem utilizando a cifra de deslocamento com $k = 3$.
  \item Criptografe essa mensagem utilizando a cifra de substituição com a seguinte permutação de letras: {\tt ZEBRASCDFGHIJKLMNOPQTUVWXY}
  \item Criptografe essa mensagem utilizando a cifra de Vigenère com chave {\tt senha}.
  \end{itemize}
\end{exercicio}

%\begin{exercicio}
  % Dá uma cifra e três chaves e pede para descriptografar?
%\end{exercicio}

\begin{exercicio}
  Mostre que a operação de adição $+$ modulo $n$ é um {\em anel} para qualquer valor de $n$.
Ou seja, para qualquer $a, b, c, n \in \mathbb{Z}$ temos que:
\begin{itemize}
\item {\em associatividade}: $(a+b)+c \equiv a+(b+c)\ mod\ n$ e $(ab)c \equiv a(bc)\ mod\ n$
\item {\em elemento neutro}: $a + 0 \equiv a\ mod\ n$ e $a.1 \equiv a\ mod\ n$
\item {\em inverso}: existe $-a$ tal que $a + (-a) \equiv 0\ mod\ n$
\item {\em distributividade}: $a(b + c) \equiv ab + ac\ mod\ n$
\end{itemize}
\end{exercicio}

\begin{exercicio}
  Mostre que se $n|a$ e $n|b$ então $n|(ra + sb)$ para quaisquer $r, s \in \mathbb{Z}$.
\end{exercicio}

\begin{exercicio}
  Dizemos que $a$ é o {\em inverso multiplicativo} de $b$ em $\mathbb{Z}_n$ sse $ab \equiv 1\ mod\ n$.
\begin{itemize}
\item Mostre que $2$ é o inverso multiplicativo de $5$ em $\mathbb{Z}_9$.
\item Mostre que $6$ não possui inverso multiplicativo em $\mathbb{Z}_{12}$
\end{itemize}
\end{exercicio}

\begin{exercicio}
  Proponha um sistema de criptografia simétrica e argumente porque ele é mais seguro do que os sistemas que vimos até aqui.
\end{exercicio}

\begin{exercicio}
Calcule o tamanho do universo das chaves em uma cifra de Vigenère da forma como usada normalmente (escolhendo um palavra) e na forma como apresentamos formalmente (sequência aleatória com tamanho fixo $l$)?
\end{exercicio}


\begin{exercicio}
Construa um script que extraia um corpus do português moderno (por exemplo, textos da wikipedia) e calcule a frequência de ocorrência das letras do alfabeto.
\end{exercicio}

\begin{exercicio}
Em 2017 um rapaz que ficou conhecido como menino do Acre ficou dias desaparecido e deixou uma serie de livros criptografados com cifra de substituição em seu quarto.
Na Figura \ref{fig:menino-do-acre} está reproduzida uma página de um desses livros.
Utilize a análise de frequência para decifrar o texto.
\end{exercicio}

\begin{figure}[htbp]
  \centering
  \includegraphics[width=.8\textwidth]{imagens/bruno.jpg}
  \caption{Texto criptografado pelo Menino do Acre}
  \label{fig:menino-do-acre}
\end{figure}
