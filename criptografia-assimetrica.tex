\chapter{Criptografia Assimétrica}
\label{cha:criptografia-assimetrica}

No capítulo anterior, vimos um protocolo que permite a troca de chaves de criptografia sem que as partes precisem se encontrar pessoalmente ou confiar em um terceiro para garantir a segurança.
Essa ideia inovadora foi apresentada pela primeira vez em um artigo importante escrito por Diffie e Hellman \cite{Diffie76}.

Nesse mesmo artigo, os autores sugeriram um conceito revolucionário: a {\em criptografia assimétrica}.
Para entender como isso funciona, imagine um cadeado.
Qualquer pessoa pode fechar o cadeado, mas que só o dono da chave consegue abri-lo.
A criptografia assimétrica é semelhante.
Cada pessoa tem duas chaves: uma chave pública, que é como o cadeado, e uma chave secreta, que é a única capaz de abri-lo.

Vamos dar um exemplo: se Alice quer enviar uma mensagem segura para Bob, ela primeiro precisa da chave pública de Bob.
Essa chave pública pode estar disponível online ou Bob pode enviá-la para Alice por qualquer meio, mesmo que não seja seguro.
Com essa chave, Alice pode ``trancar'' a mensagem, ou seja, criptografá-la, e só Bob, com sua chave secreta, conseguirá ``abrir'' a mensagem e lê-la.

O processo é ilustrado no diagrama a seguir:


\begin{center}
\begin{tikzpicture}[node distance=2cm,auto,>=latex]
\node (alice) at (0, 2){Alice};
\node (bob) at (10, 2) {Bob};
\node (eva) at (5, 2) {Eva};

\node (m1) at (0,1) {$m$};
\node (pk1) at (0,-2) {$pk$};
\node (E)  at (2,0) {$E(pk,m) = c$};
\node (D)  at (8,0) {$D(sk,c) = m$};
\node (sk) at (10,-1) {$sk$};
\node (pk2) at (10,-2) {$pk$};
\node (m2) at (10,1) {$m$};

\path[->] (eva) edge (5,1);
\draw[->] (m1) -> (E);
\draw[->] (pk1) -> (E);
\draw[->] (D) -> (m2);
\draw[->] (k2) -> (D);
\draw[->] (pk2) -> (pk1);
\draw[->] (E) -> node[above]{$c$} (D);
\end{tikzpicture}
\end{center}

Um \textit{sistema de criptografia assimétrico} é composto por três algoritmos: $Gen$, $E$ e $D$.

\begin{itemize}
\item $Gen$ é o algoritmo responsável por gerar as chaves.
  Ele cria um par de chaves: uma chave secreta, que chamamos de $sk$, e uma chave pública, que chamamos de $pk$.
\item $E$ é o algoritmo de criptografia, que usa a chave pública $pk$ para transformar uma mensagem, $m$, em uma mensagem cifrada.
\item $D$ é o algoritmo de descriptografia, que usa a chave secreta $sk$ para transformar a mensagem cifrada de volta na mensagem original.
\end{itemize}

Para garantir que o sistema seja {\em correto}, é preciso que, ao descriptografar com a chave secreta uma mensagem que foi criptografada com a chave pública, o resultado seja exatamente a mensagem original:

\begin{displaymath}
  D(sk, E(pk, m)) = m
\end{displaymath}

Isso significa que, se você criptografar uma mensagem com a chave pública de alguém e depois descriptografá-la com a chave secreta dessa pessoa, você recupera a mensagem original.

Um sistema de criptografia assimétrico é considerado \textit{seguro} se for inviável para um adversário, que tenha acesso à chave pública, descobrir qualquer informação útil sobre duas mensagens diferentes a partir da versão criptografada de uma delas.
Ou seja, mesmo que a chave pública esteja disponível para todos, ela não deve permitir que alguém decifre ou entenda a mensagem sem a chave secreta correspondente.

Em termos técnicos, ter a chave pública é semelhante a ter acesso a um oráculo, um recurso que ajuda a criptografar mensagens, mas não ajuda a decifrá-las.
Em outras palavras, a posse da chave pública dá ao adversário uma capacidade equivalente ao que assumimos em um ataque do tipo CPA.



\section{El Gammal}
\label{sec:el-gammal}

O primeiro sistema de criptografia assimétrica foi descoberto por Rivest, Shamir e Adleman, um ano após o trabalho de Diffie e Hellman.
Antes de introduzirmos essa solução, vamos apresentar um esquema que tem muitas semelhanças com o protocolo discutido no capítulo anterior.
Apesar de tantas semelhanças, este sistema só foi formalizado em 1985 por Taher El Gamal \cite{ElGamal85}.

Esse sistema de criptografia segue o modelo de criptografia assimétrica, que utiliza dois tipos de chaves: uma pública e uma privada.
Ele é composto por três algoritmos:

\begin{itemize}
\item $Gen$ cria um grupo cíclico de tamanho $n$ com um gerador $g$.
  A chave secreta $sk$ é um número $x$ menor do que $n$.
  A chave pública é $h = g^x$.
\item $E$ sorteia um número $y$ menor do que $n$ e multiplica a mensagem $m$ por $h^y$ e junta o valor $g^y$.
  \begin{displaymath}
    E(pk, m) = h^y \cdot m, g^y
  \end{displaymath}
\item $D$ pega a primeira parte da cifra ($h^y \cdot m$) e divide pela segunda parte ($g^y$) elevado a $x$.
  \begin{displaymath}
    (h^y \cdot m) \cdot (g^{xy})^{-1}
  \end{displaymath}
\end{itemize}

Não é difícil verficar que este sistema é correto.
Basta lembrar que $h = g^x$, substituir e desenvolver:

\begin{eqnarray*}
  (h^y \cdot m) \cdot (g^{xy})^{-1}) & = & (g^{xy} \cdot m) \cdot (g^{xy})^{-1})\\
                                   & = & m
\end{eqnarray*}
  
O sistema de El Gamal é seguro se, no grupo escolhido, o problema do logaritmo discreto for difícil de resolver.

\section{RSA}
\label{sec:rsa}

Em 1978, Rivest, Shamir e Adleman apresentaram o primeiro sistema de criptografia assimétrica, conforme idealizado por Diffie e Hellman um ano antes \cite{Rivest78}.
Diferente do protocolo de Diffie-Hellman e do sistema de El Gamal, que se baseiam na dificuldade do problema do logaritmo discreto, o sistema RSA depende da dificuldade de outro problema matemático: o problema da fatoração.

Antes de explicar como o sistema funciona, precisamos fazer uma breve pausa para discutir alguns conceitos matemáticos e mostrar um exemplo de grupo finito.

Vamos considerar o conjunto $\mathbb{Z}_n$ composto pelos números que vão de $0$ até $n-1$.
As operações de soma e multiplicação dentro deste conjunto serão obtidas somando os números e depois tomando o módulo por $n$.
Isso garante que toda operação permanece dentro de $\mathbb{Z}_n$.

\begin{proposition}
\label{prop:inverso}
Um número $a$ possui inverso multiplicativo em $\mathbb{Z}_n$ se e somente se $a$ e $n$ são primos entre si, ou seja $mdc(a,n) = 1$.
\end{proposition}


Se $a$ e $n$ são primos entre si, então é possível mostrar que existem números inteiros $x$ e $y$ tais que $ax + ny = 1$ (isso é conhecido como a identidade de Bezout).
Nesse caso, $x \times a = 1$ no sistema módulo $n$, ou seja, $x$ é o inverso de $a$.

Por outro lado, se $a$ e $n$ não são primos entre si, então existe um número $b > 1$ que divide tanto $a$ quanto $n$.
Agora, imagine que $a$ tivesse um inverso, ou seja, um valor $x$ tal que $ax$ é igual a 1 no sistema módulo $n$.
Isso implicaria que poderíamos escrever 1 como $ax + ny$ para algum inteiro $y$.
No entanto, como $b$ divide tanto $a$ quanto $n$, $b$ também teria que dividir 1.
Mas isso é impossível, porque o único divisor de 1 é ele mesmo, e $b > 1$. Concluímos, então, que esse $x$ não pode existir e, portanto, $a$ não tem inverso em $\mathbb{Z}_n$.

\begin{proposition}[Identidade de Bezout]
  Para quaisquer dois números inteiros $a$ e $b$ sempre existem outros dois números inteiros $x$ e $y$ tais que $Xa + Yb = mdc(a,b)$.
\end{proposition}

Considere o conjunto formado por todos os números que podem ser escritos na forma $x'a + y'b$ para $x'$ e $y'$ inteiros.
Seja $d = xa + yb$ o menor desses números que podem ser escritos dessa forma.

Agora, pegue um número qualquer $c$ que também possa ser escrito como $c = x'a + y'b$.
Se dividirmos $c$ por $d$, teremos um quociente $q$ e um resto $r$, ou seja, $c = qd + r$.

Podemos então reescrever $r$ como
\begin{displaymath}
r = c - qd = x'a + y'b - q(xa + yb) = a(x' - qx) + b(y' - qy).
\end{displaymath}

Isso mostra que o resto $r$ também pode ser escrito na forma $ax' + by'$.

Agora, o resto $r$ teria que ser menor do que $d$, mas isso contradiz o fato de que $d$ é o menor número que pode ser escrito na forma $xa + yb$.
Concluímos, então, que $r = 0$, ou seja, não há resto.

Como o resto é 0, isso significa que $d$ divide $c$.
Mas $c$ foi escolhido como qualquer número que pode ser escrito como $x'a + y'b$.
Em particular, isso vale para $a$ e para $b$.
Portanto, $d$ divide tanto $a$ quanto $b$.

Agora, resta mostrar que $d$ é o maior dos divisores de $a$ e $b$.
Pegue qualquer divisor $d'$ de $a$ e de $b$.
Nesse caso, $d'$ também dividiria $ax + by = d$.
Portanto, $d'$ teria que ser menor ou igual a $d$.

Concluímos, então, que $d$ é o maior dos divisores de $a$ e $b$, ou seja, $d = ax + by = \text{mdc}(a,b)$.

Para encontrar os coeficientes $x$ e $y$ que satisfazem a Identidade de Bezout, ou seja, $xa + yb = \text{mdc}(a, b)$, podemos utilizar uma versão estendida do algoritmo de Euclides.
Este algoritmo não apenas calcula o {\em máximo divisor comum} (mdc) de dois números $a$ e $b$, mas também determina os valores de $x$ e $y$ que, quando combinados com $a$ e $b$, resultam no mdc.

\begin{codebox}
\Procname{$\proc{AEE}(a, b)$}
\li \Comment Recebe $a, b$ inteiros com $a > b$
\li \Comment Devolve $t$, $x$ e $y$ tais que $t = mdc(a,b) = xa + yb$
\li \If $b|a$
\li     \Then
        \Return $b, 0, 1$
\li \Comment $a = bq + r$
\li \Else $t, x, y \gets \proc{AEE}(b, r)$
\li     \Return $t, y, x - y \cdot q$
\End
\end{codebox}

Vamos definir um conjunto $\mathbb{Z}_n^\star$ composto por todos os números inteiros positivos menores do que $n$ que possuem inverso multiplicativos:

\begin{displaymath}
  \mathbb{Z}_n^\star := \{a \in \mathbb{Z}_n : mdc(a,n) = 1\}
\end{displaymath}

Para todo $a$ em $\mathbb{Z}_n^\star$ temos que $a$ e $n$ são primos entre si.
Então podemos usar Algoritmo Extendido de Euclides para calcular $x$ e $y$ tais que $ax + ny = 1$.
O valor $x$ assim calculado é o inverso de $a$ em $\mathbb{Z}_n$.

Todos os elementos de $\mathbb{Z}_n^\star$ possuem inverso multiplicativo.
Não é difícil mostrar que as outras propriedades dos grupos também são satisfeitas nesse conjunto quando aplicamos a multiplicação módulo $n$ (Exercício \ref{ex:euler}).
Chamamos de {\em função de Euler} a função que nos dá o tamanho de $\mathbb{Z}_n^\star$ para cada $n$:

\begin{displaymath}
  \phi(n) := |\mathbb{Z}_n^\star|
\end{displaymath}

Quando $n$ é o produto de dois número primos, calcular a função de Euler é bastante simples:

\begin{proposition}
Sejam $p$ e $q$ números primos:
\begin{displaymath}
  \phi(pq) = (p - 1)(q - 1)
\end{displaymath}
\end{proposition}

Para entender por que essa fórmula funciona, vamos considerar o seguinte:

Se pegarmos um número $a$ dentro do conjunto $\mathbb{Z}_n$, e $a$ não tem $p$ ou $q$ como divisor, então $a$ será ``primo'' em relação a $n$.
Mas, se $a$ for divisível por $p$ ou $q$, então $a$ não será ``primo'' em relação a $n$.

Os números em $\mathbb{Z}_n$ que são divisíveis por $p$ são: $p, 2p, \dots, (q-1) \cdot p$.
Os números que são divisíveis por $q$ são: $q, 2q, \dots, (p-1) \cdot q$.

Para encontrar $\phi(n)$, subtraímos do total $(n-1)$ os números divisíveis por $p$ e os divisíveis por $q$.
O resultado nos dá a quantidade de números em $\mathbb{Z}_n$ que não são divisíveis por $p$ ou $q$, que é exatamente $(p-1)(q-1)$.

Portanto, a fórmula $\phi(pq) = (p-1)(q-1)$ nos dá o número de elementos em $\mathbb{Z}_n$ que possuem um inverso multiplicativo, ou seja, que são ``primos'' em relação a $n$.

A geração de chaves $Gen$ no sistema RSA segue os seguintes passos:

\begin{enumerate}
\item Escolhemos aleatoriamente dois números primos, $p$ e $q$, ambos com o número de bits predeterminado e suficientemente grande.
    
\item Em seguida, calculamos $N$, que é o produto dos dois primos $p \times q$.
\item Calculamos $\phi(N) = (p - 1) \times (q - 1)$. 
    
\item Agora, escolhemos um número $e$ que seja maior que 1 e que não tenha divisores comuns com $\phi(N)$, exceto o número 1.
  Isso garante que $e$ e $\phi(N)$ sejam ``primos entre si''.
    
\item Depois, calculamos $d$, que é o inverso de $e$ em relação a $\phi(N)$.
\item $N$ é a chave pública e $d$ é a chave privada.
\end{enumerate}

Na prática, podemos escolher $e$ previamente e calcular $d$ com base nisso.
Para tornar a criptografia mais eficiente, é útil escolher um valor para $e$ que tenha poucos bits {\tt 1}, o que facilita o processo de cálculo durante a criptografia.
Dois valores comumente usados para $e$ são $3$ e $2^{16} + 1$.

Como vimos, o cálculo de $d$ é feito usando o Algoritmo Estendido de Euclides.



Em sua versão crua, o sistema de RSA é definido pela seguinte tripla de algoritmos:

\begin{itemize}
\item Para criptografar uma mensagem $m$ elevamos a $e$ e calculamos o móulo po $N$. 
\item Para descriptografar elevamos a cifra $c$ por $d$ e calculamos o módulo por $N$.
\end{itemize}

Como $\mathbb{Z}_n^\star$ com a operação de multiplicação forma um grupo, segue do Teorema \ref{theo:gen-euler} que para qualquer $a$ temos que:
\begin{displaymath}
  a^{\phi(n)} \equiv 1\ (mod\ n)
\end{displaymath}

Esse resultado é chamado de {\em Teorema de Euler}.

\begin{corollary}
  Para todo $a \in \mathbb{Z}_n^\star$ temos que $a^x \equiv a^{[x\ mod\ \phi(n)]}\ (mod\ n)$.
\end{corollary}
\begin{proof}
  Seja $x = q \cdot \phi(n) + r$ em que $r = [x\ mod\ \phi(n)]$:
  \begin{displaymath}
    a^x \equiv a^{q \cdot \phi(n) + r} \equiv a^{q \cdot \phi(n)} \cdot a^r \equiv (a^{\phi(n)})^q \cdot a^r \equiv 1^q \cdot a^r \equiv a^r\ (mod\ n)
  \end{displaymath}
\end{proof}


\begin{corollary}
\label{cor:euler}
Seja $x \in \mathbb{Z}_n^\star$, $e > 0$ tal que $mdc(e, \phi(n)) = 1$ e $d \equiv e^{-1}\ (mod\ \phi(n))$ então $(x^e)^d \equiv x\ (mod\ n)$
\end{corollary}

\begin{proof}
  \begin{displaymath}
    x^{ed} \equiv x^{[ed\ mod\ \phi(n)]} \equiv x\ (mod\ n)
  \end{displaymath}
\end{proof}

Esses são os resultados necessários para mostrar a corretude do sistema RSA:
\begin{displaymath}
D(sk, E(pk, m)) = [(m^e)^d\ mod\ N] = [m^{ed}\ mod\ N] = [m\ mod\ N] = m
\end{displaymath}

Uma condição necessária para a segurança dos sistemas RSA é a dificuldade do {\em problema da fatoração}.

No problema da fatoração o adversário recebe $N = pq$ em que $p$ e $q$ são primos e deve achar os valores $p$ e $q$.

Esse problema é considerado difícil porque não conhecemos algoritmo eficiente que o resolva com probabilidade considerável.

Caso um adversário eficiente fosse capaz de fatorar $N$, ele produziria $p$ e $q$ e seria então capaz de gerar $\phi(N) = (p-1)(q-1)$ sem nenhuma dificuldade.
Com isso, o sistema se torna completamente inseguro.

Ou seja, a dificuldade do problema da fatoração é necessária para garantir a segurança do sistema. 
Não sabemos, porém, se essa condição é suficiente.

A construção apresentada acima é chamada de RSA simples ({\em plain RSA}) e é insegura por uma série de motivos.
Um mecanismo para torná-la mais segura é sortear uma sequência de bits $r$ aleatoriamente e usar $m' = r||m$ como mensagem.
O mecanismo segue de maneira idêntica, a única diferença é que, ao decifrar, se recupera $r||m$ e então é preciso descartar os primeiros bits.

Essa versão é chamada {\em padded RSA} e uma adaptação dela é usada no padrão {\tt RSA PKCS \#1 v1.5}, que é usada na prática.


\section{Sistemas Híbridos}
\label{sec:sistemas-hibridos}


Os sistemas de criptografia assimétricos que vimos até agora são, pelo menos, uma ordem de grandeza menos eficientes do que os sistemas de criptografia simétrica que estudamos anteriormente.
Além disso, é difícil garantir a segurança desses sistemas para mensagens com qualquer distribuição de probabilidades.

Por esses motivos, o modelo mais popular de criptografia assimétrica segue uma abordagem híbrida, dividida em duas fases:
uma fase de encapsulamento de chave (KEM) e, em seguida, uma fase de encapsulamento dos dados (DEM).

Um {\em mencanismo de encapsulamento de chave} (KEM) é um sistema formado por três algoritmos:
\begin{itemize}
\item $Gen$ gera um par de chaves $pk$ é uma chave pública e $sk$ uma chave secreta
\item $Encaps$ utiliza chave pública $pk$ para produzir uma cifra $c$ e uma chave $k$
\item $Decaps$ utiliza chave secreta $sk$ e a cifra $c$ para recupera a chave $k$.
\end{itemize}

Um sistema KEM é correto se o processo de decapsulamento sempre recuperar a mesma chave que foi criada na fase de encapsulamento.


Um sistema KEM é {\em seguro contra CPA} se um adversário polinomial não é capaz de distinguir com probabilidade considerável a chave produzida por $Encaps$ de uma chave de mesmo tamanho escolhida aleatoriamente.

Se $\Pi_K = \langle Gen_K, Encaps, Decaps \rangle$ é um KEM seguro contra CPA, e $\Pi' = \langle Gen', E', D' \rangle$ é um sistema de criptografia simétrica seguro contra ataques ``ciphertext only'', então o seguinte sistema, chamado de {\em híbrido}, é seguro contra CPA:

\begin{itemize}
\item $Gen$ gera um par de chaves $sk$ e $pk$
\item $E$ utiliza $pk$ para gerar duas cifras $c_k$ e $c$ da seguinte forma:
  \begin{itemize}
  \item $Encaps$ utiliza $pk$ para gerar a cifra $c_k$ e a chave $k$
  \item $E'$ utiliza $k$ para criptografar a mensagem $m$ e gerar a cifra $c$
  \end{itemize}
\item $D$ utliza $sk$ para recuperar a mensagem da seguinte forma:
  \begin{itemize}
  \item $Decaps$ utiliza a chave secreta $sk$ para descriptografar $c_k$ e recuperar a chave $k$
  \item $D'$ utiliza a chave $k$ para descriptografar a cifra $c$ e recuperar a mensagem $m$
  \end{itemize}
\end{itemize}

Uma forma natural de produzir um KEM é produzir uma chave $k$ e usar um sistema de criptografia assimétrica para criptografar a chave.
Esse esquema é seguro desde que o sistema de ciptografia assimestrica seja seguro, mas existem sistemas mais eficientes em determinados casos.
Não trataremos desses sistemas neste livro, porém.


\section{Exercícios}
\label{sec:exercicios}


\begin{exercicio}
  \label{ex:euler}
  Mostre que para qualquer $n \in \mathbb{Z}$ a estrutra $\langle \mathbb{Z}_n^\star, \cdot \rangle$ é um grupo.
\end{exercicio}


%\begin{exercicio}
  % sobre a distribuição dos primos
%\end{exercicio}
