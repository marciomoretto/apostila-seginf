\chapter*{Agradecimentos}

Gostaria primeiramente de agradecer à minha eterna orientadora, Renata Wasserman, por toda a inspiração e carinho.
Sou profundamente grato por sua dedicação e apoio durante uma fase crucial da minha formação, que me permitiu tornar-me professor universitário -- uma função que exerço com muito orgulho.
Através dela, estendo meus agradecimentos a todos os meus professores de matemática e computação, especialmente ao saudoso Gubi.

Agradeço também aos colegas do curso de Sistemas de Informação, em especial aos coordenadores, que me permitiram oferecer repetidas vezes a disciplina de Segurança da Informação.
Essas oportunidades foram fundamentais para o desenvolvimento e o aprimoramento contínuo deste material.
Um agradecimento especial ao professor Valdinei Freire, que em algumas ocasiões ofereceu a disciplina e, inclusive, utilizou este material em suas aulas, e ao prfessor Marcelo Ventura que leu e comentou uma das primeiras versões da apostila.

Agradeço também aos organizadores da Criptorrave, que mantêm viva a ideia de que a computação pode ser uma ferramenta na garantia de direitos civis, contribuindo para a construção de uma sociedade mais justa, igualitária e livre.
Meus agradecimentos vão, em especial, aos membros do antigo Saravá, da Escola de Ativismo, e à Júlia Raíces, que foi uma companhia agradável nos primeiros cursos que fiz e, posteriormente, como aluna no curso de verão.

Gostaria ainda de expressar minha gratidão aos pesquisadores do GPoPAI, especialmente àqueles que participaram do projeto de Privacidade e Vigilância.
Agradeço também aos colegas Pablo Ortellado e Ester Rizzi, que, embora não sejam da área, me incentivaram a transformar minhas notas de aula em um livro publicável.

Por fim, e o mais importante, agradeço aos alunos que participaram tanto do curso de verão quanto das diversas edições dos cursos de graduação.
Suas contribuições, dúvidas e sugestões foram inestimáveis para o aprimoramento contínuo destas notas, que agora organizo em formato de livro.
