\chapter{Integridade e Autenticidade}
\label{cha:mac}

Até agora, exploramos sistemas de criptografia simétricos que têm como principal objetivo garantir a confidencialidade da comunicação entre as partes.
No cenário mais poderoso de ataque que estudamos, o adversário tem a capacidade de escolher mensagens e verificar como elas seriam cifradas.
Também discutimos sistemas que são seguros contra esse tipo de ataque.

Neste capítulo, vamos expandir nossa discussão para abordar outros dois problemas essenciais em criptografia:

\begin{itemize}
\item[] {\em Integridade:} Como garantir que a mensagem recebida não foi alterada durante a transmissão.
\item[] {\em Autenticidade:} Como garantir que a mensagem foi realmente enviada por quem deveria enviá-la.
\end{itemize}

A importância dessas garantias é clara.
Imagine, por exemplo, uma comunicação entre um cliente e seu banco.
O banco envia uma mensagem cobrando uma dívida de um terceiro, e o cliente autoriza um pagamento de R\$1000,00.
É essencial que o cliente tenha a certeza de que a mensagem realmente veio do banco e não é um golpe (autenticidade).
Além disso, tanto o cliente quanto o banco precisam ter certeza de que ninguém alterou o valor autorizado (integridade).

Um erro comum é pensar que os sistemas de criptografia que discutimos até agora também garantem a integridade de uma mensagem.
Na realidade, esses sistemas não oferecem nenhuma proteção contra modificações no conteúdo da mensagem, e alguns deles são até {\em maleáveis}.
Isso significa que a cifra pode ser facilmente alterada por um adversário para produzir um efeito desejado.

\begin{example}
Vamos considerar um exemplo para ilustrar como um adversário pode injetar uma mensagem $m'$ em uma cifra.
Suponha que o adversário conheça a mensagem original $m$ e queira enganar o destinatário, fazendo-o acreditar que a mensagem recebida é $m'$.

O adversário pode interceptar a cifra $c$, que foi gerada pela operação $c = G(k) \xor m$, e substituí-la por uma nova cifra $c'$, calculada da seguinte forma:

\begin{displaymath}
c' = c \xor m \xor m' = G(k) \xor m \xor m \xor m' = G(k) \xor m'
\end{displaymath}

Note que, quando o destinatário descriptografar $c'$ usando a chave $k$, ele vai recuperar $m'$, exatamente como o adversário planejou.

Embora a cifra $c$ seja perfeitamente segura do ponto de vista da confidencialidade, ela não garantiu a integridade da mensagem.
Isso permitiu ao adversário alterar o conteúdo da mensagem mesmo sem que ele possuisse a chave.
\end{example}

\begin{example}
  Outro erro comum é tentar garantir a integridade de uma mensagem cifrada usando um checksum.
  Um {\em checksum}\footnote{Estudaremos isso no Capítulo \ref{cha:hash}} é como uma ``assinatura digital'' da mensagem, gerada por uma função que produz uma sequência de bits quase única para cada mensagem.
(Vamos explorar isso em mais detalhes no próximo capítulo.)

A ideia seria a seguinte: você gera um checksum para a mensagem $m$, anexa esse checksum ao final da mensagem, e só então criptografa tudo.
A lógica por trás disso é que, ao receber a cifra, o destinatário pode descriptografar, recuperar a mensagem e seu checksum, e verificar se o checksum corresponde à mensagem.
Se alguém alterar a mensagem durante o envio, o checksum não corresponderá mais, e o destinatário poderá perceber que houve uma alteração e descartar a mensagem.

Embora isso pareça uma boa ideia, ela não oferece proteção contra um adversário mal-intencionado.
Se um adversário consegue alterar a mensagem $m$, ele também pode alterar o checksum correspondente, enganando o destinatário.
Checksums funcionam bem para detectar erros acidentais, como pequenos ruídos que possam corromper a mensagem durante a transmissão, mas não são eficazes contra ataques intencionais.

\end{example}

\section{Código de Autenticação de Mensagem}
\label{sec:mac}

Para garantir que uma mensagem não foi alterada, usamos um {\em Código de Autenticação de Mensagem} (CAM).
Um CAM consiste em três algoritmos:

\begin{itemize}
\item[] {\em Geração de chave $Gen$}: Este algoritmo cria uma chave secreta, que é como uma senha especial que será usada para criar e verificar o código de autenticação.
\item[] {\em Criação de código $Mac$}: recebe uma chave $k$ e uma mensagem $m$ e devolve um código ({\em tag}) $t$.
\item[] {\em Verificação de código $Ver$}: receve uma chave $k$, uma mensagem $m$ e um código $t$ e verifica se mensagem é váilda.
\end{itemize}

Um sistema de CAM é considerado {\em correto} quando, ao usar a mesma chave secreta tanto para gerar o código quanto para verificá-lo, a verificação confirma que o código corresponde à mensagem original.
Em outras palavras, se você usa a mesma chave para criar o código e depois verificar esse código com a mensagem original, o algoritmo de verificação deve confirmar que a mensagem não foi alterada, garantindo assim a integridade da mensagem.

Vamos imaginar um cenário em que um adversário tem acesso a um sistema CAM e pode pedir a esse sistema para gerar códigos de autenticação para várias mensagens à sua escolha.
O objetivo do adversário é criar uma nova mensagem, diferente das que ele já pediu, junto com um código que seja aceito como válido pelo sistema.

Um sistema CAM é considerado {\em seguro contra falsificação} se, mesmo com todo esse acesso e poder de cálculo, o adversário não conseguir criar um par de mensagem e código que seja aceito, a não ser com probabilidade desprezível.

Vamos descrever como isso funciona:
\begin{enumerate}
\item O sistema gera uma chave.
\item O adversário pode consultar o sistema pedindo que gere códigos de autenticação para várias mensagens.
\item O desafio do adversário é criar uma nova mensagem, que não foi consultada, e inventar um código de autenticação que o sistema aceite como válido para essa mensagem.
\end{enumerate}

\begin{center}
\begin{tikzpicture}[node distance=2cm,auto,>=latex]
\tikzset{
  player/.style={draw,shape=rectangle,rounded corners,minimum width=4em,minimum height=6em}
}
\node[player,draw=none] (adversary) {\stickfigure};
\node[player] (system) at (10,0) {$\Pi$};
\draw[->] (1,0.5) -> node[above]{$q_0, \dots, q_{n'-1}$} (9,0.5);
\draw[->] (9,-0.5) -> node[above]{$Mac(k, q_0), \dots, Mac(k, q_{n'-1})$} (1,-0.5);
\draw[->] (adversary) -> node{$m, t$} (0,-2);
\end{tikzpicture}
\end{center}

Um sistema de autenticação de mensagem, é considerado {\em seguro contra falsificação} se, para qualquer adversário eficiente, a chance de ele conseguir criar um código válido para uma nova mensagem é desprezível.

\subsection{EBC-CAM}
\label{sec:cbc-mac}

Agora que entendemos o sistema de autenticação de mensagem e definimos uma noção de segurança, o próximo passo é mostrar como construir concretamente tal sistema, detalhando as suposições que faremos.

Começaremos com uma construção básica, que é válida para mensagens de tamanho fixo, utilizando o modo EBC ({\em Encadeamento de Bloco Cifrado}).

Vamos supor que o tamanho da mensagem $m$ seja um múltiplo de $n$, o tamanho do bloco de uma função pseudoaleatória (FPA) que chamaremos de $f$.
Podemos dividir a mensagem $m$ em blocos $m_1, m_2, \dots, m_l$, onde cada bloco $m_i$ tem tamanho $n$.

A construção funciona da seguinte maneira:

{\em Mac:} Este algoritmo gera um código de autenticação $t$ para a mensagem $m$ usando a chave $k$.

Primeiro, definimos $t_0$ como um bloco de zeros de tamanho $n$.
Em seguida, para cada bloco $m_i$ da mensagem, calculamos $t_i$ usando a função $f$ e a operação XOR, que combina o resultado do bloco anterior com o bloco atual:
\begin{displaymath}
t_i := f_k(t_{i-1} \oplus m_i)
\end{displaymath}

O resultado final $t_l$ é o código de autenticação.

{\em Ver:} Este algoritmo verifica se o código $t$ corresponde à mensagem $m$ quando gerado pela mesma chave $k$.

\begin{displaymath}
  Ver(k, m, t) :=  \left\{
  \begin{array}{lcl}
    1 & \textrm{se} & t = Mac(k,m)\\
    0 & \textrm{c.c.} &\\
  \end{array}
  \right.
\end{displaymath}


\begin{center}
\begin{tikzpicture}[node distance=2cm,auto,>=latex]
  \node (m1) at (0,0)  {$m_1$};
  \node (m2) at (2,0)  {$m_2$};
  \node (xor2) at (2,-2)  {$\xor$};
  \node (m3) at (4,0) {$m_3$};
  \node (xor3) at (4,-2)  {$\xor$};
  \node at (6,0) {\dots};
  \node at (6,-2) {\dots};
  \node (ml) at (8,0) {$m_l$};
  \node (xorl) at (8,-2)  {$\xor$};
  \node (t) at (10,-2)  {$t$};
  \draw[-] (m1) -> (0,-2);
  \draw[->] (0,-2) -> node[above]{$f_k$}(xor2);
  \draw[->] (m2) -> (xor2);
  \draw[->] (xor2) -> node[above]{$f_k$}(xor3);
  \draw[->] (m3) -> (xor3);
  \draw[->] (ml) ->  (xorl);
  \draw[->] (7,-2) -> node[above]{$f_k$}(xorl);
  \draw[->] (xorl) ->  (t);
\end{tikzpicture}
\end{center}


\begin{theorem}
  O sistema de autenticação EBC é seguro contra falsificação para mensagens de tamanho fixo.
\end{theorem}

O EBC-CAM é uma técnica específica para gerar códigos de autenticação.
Ele funciona de forma semelhante ao modo de operação EBC (Cipher Block Chaining), mas com duas diferenças importantes:

\begin{enumerate}
\item No EBC, usamos um vetor inicial (VI) aleatório para começar a cifrar os blocos da mensagem.
  Já no EBC-CAM, esse vetor inicial é substituído por um bloco cheio de zeros.
\item No EBC, cada bloco cifrado da mensagem é exposto, enquanto no EBC-CAM, apenas o último bloco cifrado (o CAM) é mostrado.
\end{enumerate}

O EBC-CAM é originalmente projetado para funcionar com mensagens de tamanho fixo, mas muitas vezes precisamos lidar com mensagens de tamanhos variados.
Para garantir a segurança nesses casos, existem duas estratégias eficazes:

\begin{enumerate}
\item Antes de calcular o CAM, adicionamos o tamanho da mensagem no começo do conteúdo.
  Isso impede que alguém consiga criar dois CAMs iguais para mensagens diferentes (por exemplo, uma mensagem curta poderia ser estendida para enganar o sistema).
\item Outra abordagem é usar duas chaves diferentes.
  A primeira chave é usada normalmente para gerar o CAM, e a segunda é usada para ``ajustar'' esse CAM final, reforçando a segurança.
\end{enumerate}

Essas técnicas garantem que o EBC-CAM permaneça seguro, mesmo quando as mensagens variam em tamanho, protegendo contra tentativas de falsificação.


\section{Criptografia Autenticada}
\label{label}

No modelo ETP (Ataque com Escolha de Texto Plano), o adversário pode acessar um oráculo que criptografa mensagens escolhidas para tentar descobrir segredos do sistema.
Agora, vamos considerar um modelo ainda mais forte de ameaça.

Além de poder enviar mensagens para serem criptografadas, o adversário também pode consultar como certas cifras seriam decifradas.
Isso é muito mais poderoso, porque o adversário não está apenas vendo as mensagens antes de serem criptografadas, mas também pode ver como qualquer cifra seria transformada de volta em texto legível.

Esse novo cenário é raro na prática, porque na maioria das situações reais, os adversários não têm esse tipo de acesso. Entretanto, saber que é possível construir sistemas que permanecem seguros mesmo nesse cenário é fascinante. Isso mostra a robustez e a força das técnicas modernas de criptografia.

Mesmo que seja difícil imaginar uma situação prática onde um adversário tenha tanto poder, explorar esses modelos nos ajuda a desenvolver sistemas de criptografia que sejam ainda mais fortes e seguros.
Além disso, entender esses cenários extremos nos dá mais confiança de que os sistemas que usamos todos os dias são realmente protegidos contra uma ampla gama de ameaças possíveis \cite{Naor90}.

Para entender se um sistema de criptografia é realmente seguro, imaginamos um ``jogo'' entre o sistema e um adversário, que é alguém tentando quebrá-lo.
O jogo funciona assim:
\begin{enumerate}
\item O adversário escolhe duas mensagens, que têm exatamente o mesmo tamanho, e as envia para o sistema.
\item O sistema gera uma chave e, em seguida, escolhe aleatoriamente uma das duas mensagens para criptografar.
  A mensagem escolhida é criptografada, e enviada para o adversário.
\item Durante o jogo, o adversário tem acesso a dois ``oráculos'':
  \begin{itemize}
  \item Um oráculo que mostra como qualquer mensagem escolhida pelo adversário seria criptografada.
  \item Outro oráculo que revela como qualquer cifra seria decifrada, exceto a cifra que foi devolvida pelo sistema.
  \end{itemize}
\item Com base em todas as informações obtidas, o adversário tenta adivinhar qual das duas mensagens originais foi criptografada.
\end{enumerate}

\begin{center}
\begin{tikzpicture}[node distance=2cm,auto,>=latex]
\tikzset{
  player/.style={draw,shape=rectangle,rounded corners,minimum width=6em,minimum height=15em}
}
\node[player,draw=none] (adversary) {\stickfigure};
\node[player] (system) at (10,0) {$\Pi$};
\draw[->] (1.2,2) -> node[above]{$m_0, m_1$} (8.8,2);
\draw[->] (1.2,1) -> node[above]{$q_0, \dots, q_{n'-1}$} (8.8,1);
\draw[->] (8.8,0) -> node[above]{$E(k, q_0), \dots, E(k, q_{n'-1})$} (1.2,0);
\draw[->] (1.2,-1) -> node[above]{$q_0', \dots q_{n'-1}'$} (8.8,-1);
\draw[->] (8.8,-2) -> node[above]{$D(k,q_1'), \dots, D(k,q_{n'-1}')$} (1.2,-2);
\draw[->] (8.8,-3) -> node[above]{$E(k,m) = c$} (1.2,-3);
\draw[->] (10,4) -> node{$m \leftarrow \{m_0,m_1\}$} (system);
\end{tikzpicture}
\end{center}

Dizemos que um sistema é seguro contra o {\em Ataque com Escolha de Texto Cifrado} (ETC) se para qualquer adversário eficiente existe a chance do adversário vencer o jogo é desprezivelmente maior que $1/2$.

Sistemas de criptografia que não garantem a integridade das mensagens não são seguros contra ataques do tipo ETC.
Um exemplo que ilustra bem essa vulnerabilidade é o modo Ctr de cifração de blocos.

\begin{example}
No modo Ctr, o adversário pode explorar a falta de integridade da seguinte maneira:
\begin{enumerate}
\item  O adversário  envia duas mensagens distintas e simples para o sistema: 
$m_0 = 0^n$ (um bloco de n bits todos iguais a 0) e $m_1 = 1^n$ (um bloco de n bits todos iguais a 1).
\item O sistema escolhe uma das mensagens, criptografa e envia a cifra correspondente $c$ de volta ao adversário.
\item O adversário altera o primeiro bit da cifra $c$ gerando uma nova cifra $c'$ que é diferente da original.
\item O adversário então usa o oráculo de decifragem para verificar o resultado da decifragem de $c'$.
  Se o resultado for $10^{n-1}$, ele deduz que o sistema criptografou $m_0 = 0^n$.
  Se o resultado for $01^{n-1}$, ele deduz que o sistema criptografou $m_1 = 1^n$.
\end{enumerate}

Neste cenário, o adversário consegue descobrir qual mensagem foi criptografada apenas alterando a cifra e observando o comportamento do sistema.
Isso evidencia a importância de proteger a integridade da mensagem para garantir a segurança contra ataques ETC.
\end{example}


Em um sistema de {\em criptografia autenticado}, a segurança vai além de simplesmente proteger o conteúdo das mensagens.
Aqui, o sistema precisa ser capaz de detectar quando uma cifra foi alterada ou criada de forma maliciosa.
Se uma cifra não for válida, o sistema deve acusar um erro e se recusar a tentar decifrá-la.

Uma característica importante de um sistema de criptografia autenticado é sua capacidade de impedir que um adversário produza uma cifra válida de maneira fraudulenta.
Se um sistema consegue garantir que qualquer tentativa de falsificação de uma cifra resulte em uma mensagem de erro, dizemos que ele é {\em seguro contra falsificação}.

Para que um sistema de criptografia seja considerado verdadeiramente seguro e autenticado, ele deve satisfazer duas condições:
\begin{itemize}
\item Ser seguro contra ataques do tipo ETC.
\item Ser seguro contra falsificação.
\end{itemize}

Quando um sistema atende a esses dois requisitos, chamamos ele de um {\em sistema autenticado de criptografia}.

Um {\em sistema autenticado de criptografia} é a junção de dois elementos essenciais:
um esquema de criptografia para proteger o conteúdo das mensagens e um sistema de autenticação para garantir que as mensagens não foram adulteradas.
Existem várias maneiras de combinar esses dois sistemas para alcançar segurança completa. Vamos explorar as principais:

\begin{itemize}
\item Autenticar e depois criptografar ({\em mac then encrypt}):
  Primeiro, geramos um código de autenticação para a mensagem.
  Em seguida, juntamos esse código com a mensagem original e criptografamos tudo de uma vez. 
\item Criptografar e autenticar separadamente ({\em encrypt and mac}):
  Neste método, primeiro criptografamos a mensagem para protegê-la.
  Em seguida, geramos um código de autenticação para a mensagem original.
  Ambos, a cifra e o código de autenticação, são enviados separadamente.
\item Criptografar e depois autenticar ({\em encrypt then mac}):
  Aqui, começamos criptografando a mensagem.
  Depois de criptografada, geramos um código de autenticação baseado na cifra resultante. 
\end{itemize}

\begin{theorem}
  O modelo que primeiro criptografa e depois autentica ({\em encrypt then mac}) que usa um sistema de criptografia seguro contra CPA e um sistema de autenticação seguro contra falsificação é um sistema de criptografia autenticada seguro contra ataques ETC e seguro contra falsificação. 
\end{theorem}

Para que o sistema acima seja seguro, é essencial que ele use uma chave para criptografar e outra distinta e independente para autenticar.

\begin{example}
Vamos explorar um exemplo onde o uso de uma única chave para criptografia e autenticação pode causar sérios problemas de segurança.

Imagine que temos uma permutação pseudoaleatória $p$.
Para criptografar uma mensagem $m$ (que tem metade do tamanho de um bloco), fazemos o seguinte:
\begin{itemize}
\item Escolhemos um valor aleatório $r$ com metade dos bits de um bloco.
\item Combinamos a mensagem $m$ e o valor $r$ e aplicamos a permutação $p$ ($E(k, m) = p_k(m || r)$).
\end{itemize}

Podemos provar que essa forma de criptografia é segura contra ataques CPA.

Agora, vamos usar a inversa da permutação $p^{-1}$ para criar um código de autenticação (CAM).
Ou seja, $t = p^{-1}(c)$.

Também podemos mostrar que esse CAM é seguro contra falsificação.

Agora, vamos ver o que acontece quando combinamos esses dois sistemas usando o paradigma \textit{Encrypt then Mac} \textbf{sem} usar chaves independentes:

\begin{itemize}
    \item Primeiro, criptografamos a mensagem: $E(k, m) = p_k(m || r)$.
    \item Em seguida, aplicamos o CAM na cifra: $Mac(k, E(k, m)) = p_k^{-1}(p_k(m || r))$.
\end{itemize}

Quando aplicamos o CAM na cifra, o resultado é a mensagem original $m$ e o valor aleatório $r$: $p_k^{-1}(p_k(m || r)) = m || r$.

Isso significa que o código de autenticação revela exatamente o conteúdo da mensagem original, que deveria estar protegido!
Em outras palavras, o adversário consegue ver a mensagem inteira simplesmente olhando para o código de autenticação, o que quebra completamente a segurança do sistema.

Esse exemplo mostra por que é importante usar chaves independentes para criptografia e autenticação.
Quando não usamos chaves diferentes, corremos o risco de comprometer a segurança do sistema, permitindo que informações confidenciais vazem de maneiras inesperadas.

\end{example}

\subsection{Comunicação Segura}
\label{sec:comunicacao-segura}

Um sistema autenticado de criptografia é responsável por garantir três aspectos importantes na comunicação:
{\em confidencialidade} (apenas o destinatário deve ser capaz de ler a mensagem), {\em integridade} (a mensagem não deve ser alterada) e {\em autenticidade} (o destinatário deve ter certeza de quem enviou a mensagem).
No entanto, mesmo com essas proteções, ainda existem outros tipos de ameaças que precisam ser considerados.
Vamos ver três dessas ameaças e como podemos lidar com elas.



Um sistema autenticado de criptografia garante a confidencialidade, a integridade e a autenticidade na comunicação.
Existem, porém, outros tipos de ameaça que esse sistema não garante por si só:
\begin{itemize}
\item[] {\em Ataques de reordenação}:
  Um adversário pode mudar a ordem das mensagens enviadas, o que pode causar problemas.
  Imagine um cenário onde as mensagens devem ser processadas em uma ordem específica;
  se a ordem for alterada, isso pode levar a ações incorretas ou até mesmo prejudiciais.

  Para evitar isso, podemos usar um contador que mantém o registro da ordem das mensagens enviadas entre dois participantes.
  Por exemplo, um contador $ctr_{AB}$ pode ser usado para as mensagens de $A$ para $B$, e $ctr_{BA}$ para as mensagens de $B$ para $A$.
  Assim, se as mensagens chegarem fora de ordem, o sistema detectará e rejeitará a comunicação.
\item[] {\em Ataque de repetição:}
  Aqui, o adversário repete uma mensagem legítima.
  Por exemplo, se alguém envia uma ordem de pagamento, o adversário pode repetir essa mensagem, tentando fazer com que o pagamento seja processado duas vezes.

  Novamente, um contador pode ajudar.
  Como cada mensagem terá um número único, o sistema saberá se uma mensagem está sendo repetida e poderá rejeitar tentativas de repetição.
\item[] {\em Ataque de reflexão:}
  Neste ataque, um adversário intercepta uma mensagem e a envia de volta ao remetente, fazendo parecer que ela veio do destinatário.
  Isso pode confundir o remetente ou fazer com que ele realize uma ação indesejada.

  Podemos adicionar um bit à mensagem que indique a direção da comunicação.
  Se o bit $b_{AB}$ estiver definido como 1, isso significa que a mensagem foi enviada de $A$ para $B$.
  Se estiver definido como 0, significa que a mensagem foi de $B$ para $A$.
  Isso impede confusões sobre quem enviou a mensagem.

  Outra solução é usar duas chaves independentes:
  uma $k_{AB}$ para mensagens de $A$ para $B$ e outra $k_{BA}$ para mensagens de $B$ para $A$.
  Dessa forma, mesmo que uma mensagem seja enviada de volta, a chave errada será usada, tornando a mensagem inválida.

Ambas as soluções -- tanto o uso de contadores quanto o bit de direção -- devem ser incorporadas na mensagem antes de ela ser criptografada.
Isso garante que a integridade e a autenticidade da mensagem sejam mantidas.
\end{itemize}

Com isso, concluímos nossa exploração dos sistemas de criptografia simétrica.
Utilizando uma abordagem moderna, explicamos as suposições necessárias para garantir a segurança em diferentes sistemas de criptografia e autenticação.

No Capítulo \ref{cha:owf}, examinamos a suposição mínima necessária para construir os sistemas que discutimos: a existência de funções de mão única.
Esse capítulo não só servirá como um resumo do que aprendemos até agora, mas também explicará por que precisamos dessas suposições básicas para provar a segurança em criptografia.

\section{Exercícios}
\label{sec:exercicios}


\begin{exercicio}
  Considere que o adversário sabe que a mensagem $m = 101010$ foi cifrada por uma cifra de fluxo e produziu $c = 110001$.
  Que sequencia de bits $c'$ ele precisa enviar para o destinatário para fazer com que ele ache que a mensagem original era $m' = 001011$?
\end{exercicio}


\begin{exercicio}
  Mostre que a função $encode$ que insere o tamanho da mensagem no começo e completa o último bloco com uma sequência de $0$s não admite prefixo.
\end{exercicio}

\begin{exercicio}
  Seja $f$ uma função pseudo-aleatória e considere o sistema $\Pi$ uma cifra de bloco que aplica $f$ no modo contador.
  Suponha que Alice e Bob compartilham uma chave secreta $k$.
  Considere os seguintes cenários:
  \begin{enumerate}
  \item Alice enviar $E(k, m)$ para Bob que descriptografa usando a chave $k$
  \item Alice gera um checksum $H(m)$ da mensagem e envia $H(m)||m$ para Bob que pode verificar o checksum antes de ler a mensagem
  \end{enumerate}
  Algum desses cenários garante que a mensagem lida por Bob é idêntica a mensagem que foi enviada por Alice? Por que? Caso nenhum dos cenários garanta isso, descreva como poderíamos fazê-lo.
\end{exercicio}
