\chapter*{Apresentação}

Esta obra nasceu como uma apostila de notas de aula para a disciplina de Segurança da Informação, ministrada no segundo semestre de 2017 para as turmas diurnas e noturnas do curso de Sistemas de Informação da Escola de Artes, Ciências e Humanidades (EACH) da USP.
A primeira versão foi preparada para o curso de verão oferecido entre 2 e 6 de fevereiro de 2015, no campus leste da USP, como parte das atividades do projeto de Privacidade e Vigilância do Grupo de Políticas Públicas em Acesso à Informação (GPoPAI).
Este curso foi inspirado pelo conteúdo do curso online gratuito do professor Dan Boneh, disponível na plataforma Coursera, que por sua vez se baseia no livro de Jonathan Katz e Yehuda Lindell, Introduction to Modern Cryptography.

Ao longo dos anos, a apostila evoluiu e foi continuamente revisada.
Ministrei regularmente a disciplina de graduação em Segurança da Informação desde 2016 até o momento em que escrevo estas linhas no segundo semestre de 2023.
As primeiras versões do material foram fortemente influenciadas pelo livro de Katz e Lindell.
Com o tempo, outras referências tornaram-se essenciais, como o curso no YouTube de Christof Paar e o livro Understanding Cryptography, escrito por ele e Jan Pelzl.
Em especial, o capítulo sobre Funções de Mão Única foi fortemente inspirado pelo livro Foundations of Cryptography, de Oded Goldreich.
As referências históricas desta obra foram retiradas de fontes como o livro de divulgação científica, O Código, de Simon Singh e a obra The Codebreakers de David Kahn, essa, mais acadêmica, traz uma análise aprofundada sobre a história da criptografia.

Ao longo dos anos, tive a oportunidade de revisar as notas com base no feedback dos alunos e na leitura gradual dos artigos originais.
Essas referências estão citadas ao longo do livro.

O projeto de Privacidade e Vigilância do GPoPAI se desenvolveu no contexto da crise de vigilância em massa revelada por Edward Snowden em 2013.
Esta crise desencadeou uma série de eventos globais, e em São Paulo deu origem às criptorraves, um evento anual que acontece desde então, promovendo discussões sobre privacidade e segurança digital.
Como participante ativo de praticamente todas as edições, essas experiências também influenciaram profundamente a minha formação na área.

Agradeço aos alunos que participaram do curso de verão e das disciplinas de graduação oferecidas de 2016 a 2023.
Suas contribuições, dúvidas e sugestões foram valiosas para o aprimoramento contínuo destas notas, que agora organizo em formato de livro.

A introdução deste livro explora a motivação por trás do projeto de Privacidade e Vigilância e das criptorraves, focando no problema da vigilância em massa realizada pela NSA e revelada por Edward Snowden em 2013, sugerimos que essa crise foi mitigada pela adoção massiva de aplicativos de comunicação com criptografia ponta a ponta, como o Signal.
Posteriormente, o mesmo protocolo de segurança foi implementado em aplicativos de larga escala, como o WhatsApp, tornando essa proteção acessível a milhões de usuários.

Após essa introdução, que o livro está dividido em duas partes.
A primeira parte aborda a criptografia simétrica, começando pelas cifras clássicas, como a cifra de deslocamento, cifra de substituição e a cifra de Vigenère, além de discutir técnicas de criptoanálise.
São discutidas as limitações dessas cifras clássicas e, como uma primeira aproximação ao paradigma moderno, introduzimos o conceito de sigilo perfeito.
Contudo, um dos resultados mais importantes da criptografia mostra que é impossível atingir o sigilo perfeito em uma cifra prática.
A partir desse desafio surgem as principais ideias da criptografia moderna.
Dentro desse contexto, o livro explora cifras de fluxo e cifras de bloco, como o Data Encryption Standard (DES) e o Advanced Encryption Standard (AES).
Também diferenciamos os problemas de integridade e autenticidade do problema da confidencialidade, levando à introdução de conceitos como códigos de autenticação e criptografia autenticada.

A primeira parte se encerra com um capítulo mais teórico sobre funções de mão única, que resume os principais resultados da primeira parte.
As funções de mão única representam a condição mínima -- nunca demonstrada formalmente -- para garantir a segurança dos sistemas de criptografia.

A segunda parte do livro se dedica à criptografia assimétrica.
Antes de explorarmos esse tema, estudamos as funções de hash, com foco em construções como o SHA-1 e em aplicações como o HMAC e funções de derivação de chaves.
Em seguida, cobrimos a distribuição de chaves, incluindo o protocolo de Diffie-Hellman e sistemas assimétricos como ElGamal e RSA.
O livro se conclui com capítulos sobre assinaturas digitais, infraestrutura de chaves públicas e uma breve introdução a protocolos populares.

O trajeto deste livro é didático e não histórico, pois a criptografia moderna tem suas raízes na necessidade de demonstrar a segurança dos sistemas assimétricos no final dos anos 1970, com os trabalhos seminais de Diffie, Hellman, e do trio Rivest, Shamir e Adleman.
Embora as definições e demonstrações formais de segurança para sistemas simétricos tenham sido desenvolvidas posteriormente, nos anos 1980, por autores como Blum, Micali, Yao e Goldreich, as cifras simétricas já eram conhecidas e usadas há muito mais tempo.
Por serem mais familiares aos estudantes, começamos o livro pelos sistemas simétricos, como cifras de fluxo e cifras de bloco, antes de abordar os sistemas assimétricos.


%Alguns direitos sobre o conteúdo desta apostila são protegidos pelo autor sob licença Creative Commons
%Attribution-NonCommercial-ShareAlike 4.0 International (CC BY-NC-SA 4.0). Ou seja, você  é livre para distribuir cópias e adaptar este trabalho desde que mantenha a mesma licença, dê o devido crédito ao autor e não faça uso comercial.

%\begin{center}
%  \includegraphics[width=.3\textwidth]{imagens/cc.png}
%\end{center}
