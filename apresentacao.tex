\chapter*{Apresentação}

Esta obra nasceu como uma apostila de notas de aula para a disciplina de Segurança da Informação, ministrada no segundo semestre de 2017 para as turmas diurnas e noturnas do curso de Sistemas de Informação da Escola de Artes, Ciências e Humanidades (EACH) da USP.
A primeira versão foi preparada para o curso de verão oferecido entre 2 e 6 de fevereiro de 2015, no campus leste da USP, como parte das atividades do projeto de Privacidade e Vigilância do Grupo de Políticas Públicas em Acesso à Informação (GPoPAI).

O curso foi inicialmente oferecido em um contexto em que se fazia urgente a sensibilização sobre a importância da adoção de ferramentas de comunicação segura.
Com as revelações de Edward Snowden, em 2013, sobre a vigilância em massa conduzida pela NSA, tornou-se evidente a necessidade de capacitar a sociedade para se proteger contra essas práticas invasivas. O objetivo do curso não era apenas educar sobre criptografia, mas também mobilizar os participantes a adotar práticas seguras em suas comunicações diárias.

A recepção positiva e o feedback dos participantes das primeiras edições do curso destacaram a demanda por uma educação prática e acessível em segurança da informação. Isso me motivou a adaptar o curso para o formato de graduação, integrando-o ao currículo de Sistemas de Informação da EACH-USP.
Desde então, o curso tem sido oferecido regularmente, com foco em apresentar primitivas criptográficas como ``blocos de construção'' para protocolos de segurança robustos.
Embora o curso seja teórico, ele foi concebido para ser acessível, com ênfase em como definir e reconhecer o que faz uma primitiva criptográfica ser considerada segura.
Estudamos diferentes definições de segurança e as suposições que sustentam essas garantias.

Ministrei regularmente a disciplina de graduação em Segurança da Informação desde 2016 até o momento em que escrevo estas linhas no segundo semestre de 2024.
As primeiras versões do material foram fortemente influenciadas pelo livro de Jonathan Katz e Yehuda Lindell, {\em Introduction to Modern Cryptography}, e pela abordagem do curso online de Dan Boneh, disponível na plataforma Coursera.
Com o tempo, outras referências se tornaram essenciais, como o curso de Christof Paar no YouTube e o livro {\em Understanding Cryptography}, escrito por ele e Jan Pelzl.
Em especial, o capítulo sobre Funções de Mão Única foi inspirado pelo livro {\em Foundations of Cryptography}, de Oded Goldreich.
As referências históricas, por sua vez, foram retiradas principalmente dos livros {\em O Código}, de Simon Singh, e {\em The Codebreakers}, de David Kahn.

Esta disciplina tem uma conexão importante com outras áreas do currículo de Sistemas de Informação, funcionando como uma conclusão que amarra temas explorados em disciplinas como análise de algoritmos, redes, matemática discreta e teoria da computação.
A disciplina foi concebida como uma aplicação prática da teoria da computação, permitindo aos alunos compreender as primitivas criptográficas no contexto real de uso e, com isso, evitar falhas que possam comprometer a segurança da informação.

Embora a disciplina e este livro não tenham como objetivo que os alunos implementem primitivas ou protocolos criptográficos, o conhecimento das garantias de segurança permite que eles compreendam os protocolos que utilizarão no futuro, garantindo assim que erros não comprometam a integridade das informações sob sua responsabilidade.

Ao longo dos anos, tive a oportunidade de revisar as notas com base no feedback dos alunos e na leitura gradual dos artigos originais.
Essas referências estão citadas ao longo do livro.
Além disso, o envolvimento com eventos como as Criptorraves, que surgiram como resposta à crise de vigilância em massa revelada por Snowden, influenciaram diretamente minha formação e as edições do curso.

É importante ressaltar que, até o momento, desconheço a existência de livros didáticos em português que abordem o tema da criptografia partindo da abordagem da criptografia moderna.
A publicação deste livro busca preencher essa lacuna, oferecendo aos estudantes e profissionais uma introdução acessível e completa sobre o tema, com foco nas bases teóricas e práticas essenciais para entender as técnicas e algoritmos usados atualmente.

A introdução deste livro explora a motivação por trás do projeto de Privacidade e Vigilância e das criptorraves, focando no problema da vigilância em massa realizada pela NSA e revelada por Edward Snowden em 2013, sugerimos que essa crise foi mitigada pela adoção massiva de aplicativos de comunicação com criptografia ponta a ponta, como o Signal.
Posteriormente, o mesmo protocolo de segurança foi implementado em aplicativos de larga escala, como o WhatsApp, tornando essa proteção acessível a milhões de usuários.

Após essa introdução, o livro está dividido em duas partes.
A primeira parte aborda a criptografia simétrica, começando pelas cifras clássicas, como a cifra de deslocamento, cifra de substituição e a cifra de Vigenère, além de discutir técnicas de criptoanálise.
São discutidas as limitações dessas cifras clássicas e, como uma primeira aproximação ao paradigma moderno, introduzimos o conceito de sigilo perfeito.
Contudo, um dos resultados mais importantes da criptografia mostra que é impossível atingir o sigilo perfeito em uma cifra prática.
A partir desse desafio surgem as principais ideias da criptografia moderna.
Dentro desse contexto, o livro explora cifras de fluxo e cifras de bloco, como o Data Encryption Standard (DES) e o Advanced Encryption Standard (AES).
Também diferenciamos os problemas de integridade e autenticidade do problema da confidencialidade, levando à introdução de conceitos como códigos de autenticação e criptografia autenticada.

A primeira parte se encerra com um capítulo mais teórico sobre funções de mão única, que resume os principais resultados da primeira parte.
As funções de mão única representam a condição mínima -- nunca demonstrada formalmente -- para garantir a segurança dos sistemas de criptografia.

A segunda parte do livro se dedica à criptografia assimétrica.
Antes de explorarmos esse tema, estudamos as funções de hash, apresentando a construção do SHA-1 e aplicações como o HMAC e funções de derivação de chaves.
Em seguida, cobrimos a distribuição de chaves, incluindo o protocolo de Diffie-Hellman e sistemas assimétricos como ElGamal e RSA.
O livro se conclui com capítulos sobre assinaturas digitais, infraestrutura de chaves públicas e uma breve introdução a protocolos populares.

O trajeto deste livro é didático e não histórico, pois a criptografia moderna tem suas raízes na necessidade de demonstrar a segurança dos sistemas assimétricos no final dos anos 1970, com os trabalhos seminais de Diffie, Hellman, e do trio Rivest, Shamir e Adleman.
Embora as definições e demonstrações formais de segurança para sistemas simétricos tenham sido desenvolvidas posteriormente, nos anos 1980, por autores como Blum, Micali, Yao e Goldreich, as cifras simétricas já eram conhecidas e usadas há muito mais tempo.
Por serem mais familiares aos estudantes, começamos o livro pelos sistemas simétricos, como cifras de fluxo e cifras de bloco, antes de abordar os sistemas assimétricos.

A disciplina que deu origem ao livro tem uma carga didática total de 60 horas.
Organizei o curso em 12 aulas de aproximadamente 4 horas cada, sendo que cada uma corresponde a um capítulo do livro, com exceção do capítulo sobre cifras de bloco, que ocupa duas aulas:
uma para as definições e modos de operação e outra para as construções práticas, como o DES e o AES.
Para completar a carga horária, incluo uma aula de apresentação do curso e algumas aulas dedicadas à correção de listas de exercícios, resolução de dúvidas e provas ao final de cada uma das duas partes.


%Alguns direitos sobre o conteúdo desta apostila são protegidos pelo autor sob licença Creative Commons
%Attribution-NonCommercial-ShareAlike 4.0 International (CC BY-NC-SA 4.0). Ou seja, você  é livre para distribuir cópias e adaptar este trabalho desde que mantenha a mesma licença, dê o devido crédito ao autor e não faça uso comercial.

%\begin{center}
%  \includegraphics[width=.3\textwidth]{imagens/cc.png}
%\end{center}
